
\documentclass[UTF8]{amsart}
%Typical documenttypes: article/book
%some examples:
%\documentclass[reqno,11pt]{book}   %%%for books
%\documentclass[]{minimal}			%%%for Minimal Working Example


%for beamers, you have to change a lot. Especially, remove the package enumitem!!!



%%%%%%%%%%%%%%%%%%%% setting for fast compiling

%\special{dvipdfmx:config z 0}		% no compression

\includeonly{chapters/chapter9}		% In practice, use an empty document called "chapter9"	% usually for printing books






%%%%%%%%%%%%%%%%%%%% here we include packages

%%%basic packages for math articles
\usepackage{amssymb}
\usepackage{amsthm}
\usepackage{amsmath}
\usepackage{amsfonts}
\usepackage[shortlabels]{enumitem}	% It supersedes both enumerate and mdwlist. The package option shortlabels is included to configure the labels like in enumerate.

%%%packages for special symbols
\usepackage{pifont}					% Access to PostScript standard Symbol and Dingbats fonts
\usepackage{wasysym}				% additional characters
\usepackage{bm}						% bold fonts: \bm{...}
\usepackage{extarrows}				% may be replaced by tikz-cd
%\usepackage{unicode-math}			% unicode maths for math fonts, now I don't know how to include it
%\usepackage{ctex}					% Chinese characters, huge difference.


%%%basic packages for fancy electronic documents
\usepackage[colorlinks]{hyperref}
\usepackage[table,hyperref]{xcolor} 			% before tikz-cd. 
%\usepackage[table,hyperref,monochrome]{xcolor}	% disable colored output (black and white)

%%%packages for figures and tables (general setting)
\usepackage{float}				%Improved interface for floating objects
\usepackage{caption,subcaption}
\usepackage{adjustbox}			% for me it is usually used in tables 
\usepackage{stackengine}		%baseline changes

%%%packages for commutative diagrams
\usepackage{tikz-cd}
\usepackage{quiver}			% see https://q.uiver.app/

%%%packages for pictures
\usepackage[width=0.5,tiewidth=0.7]{strands}
\usepackage{graphicx}			% Enhanced support for graphics

%%%packages for tables and general settings
\usepackage{array}
\usepackage{makecell}
\usepackage{multicol}
\usepackage{multirow}
\usepackage{diagbox}
\usepackage{longtable}

%%%packages for ToC, LoF and LoT







 %https://tex.stackexchange.com/questions/58852/possible-incompatibility-with-enumitem










%%%%%%%%%%%%%%%%%%%% here we include theoremstyles

\numberwithin{equation}{section}

\theoremstyle{plain}
\newtheorem{theorem}{Theorem}[section]

\newtheorem{setting}[theorem]{Setting}
\newtheorem{definition}[theorem]{Definition}
\newtheorem{lemma}[theorem]{Lemma}
\newtheorem{proposition}[theorem]{Proposition}
\newtheorem{corollary}[theorem]{Corollary}
\newtheorem{conjecture}[theorem]{Conjecture}

\newtheorem{claim}[theorem]{Claim}
\newtheorem{eg}[theorem]{Example}
\newtheorem{ex}[theorem]{Exercise}
\newtheorem{fact}[theorem]{Fact}
\newtheorem{ques}[theorem]{Question}
\newtheorem{warning}[theorem]{Warning}



\newtheorem*{bbox}{Black box}
\newtheorem*{notation}{Conventions and Notations}


\numberwithin{equation}{section}


\theoremstyle{remark}

\newtheorem{remark}[theorem]{Remark}
\newtheorem*{remarks}{Remarks}

%%% for important theorems
%\newtheoremstyle{theoremletter}{4mm}{1mm}{\itshape}{ }{\bfseries}{}{ }{}
%\theoremstyle{theoremletter}
%\newtheorem{theoremA}{Theorem}
%\renewcommand{\thetheoremA}{A}
%\newtheorem{theoremB}{Theorem}
%\renewcommand{\thetheoremB}{B}







%%%%%%%%%%%%%%%%%%%% here we declare some symbols

%%%%%%%DeclareMathOperator
%see here for why newcommand is better for DeclareMathOperator: https://tex.stackexchange.com/questions/67506/newcommand-vs-declaremathoperator

%%%%%basic symbols. Keep them!

%%%symbols for sets and maps
\DeclareMathOperator{\pt}{\operatorname{pt}}	%points. Other possibilities are \{pt\}, \{*\}, pt, * ...
\DeclareMathOperator{\Id}{\operatorname{Id}}	%identity in groups.
\DeclareMathOperator{\Img}{\operatorname{Im}}

\DeclareMathOperator{\Ob}{\operatorname{Ob}}
\DeclareMathOperator{\Mor}{\operatorname{Mor}}	%difference of Mor and Hom: Hom is usually for abelian categories
\DeclareMathOperator{\Hom}{\operatorname{Hom}}	\DeclareMathOperator{\End}{\operatorname{End}}
\DeclareMathOperator{\Aut}{\operatorname{Aut}}

%%%symbols for linear algebras and 
%%linear algebras
\DeclareMathOperator{\tr}{\operatorname{tr}}
\DeclareMathOperator{\diag}{\operatorname{diag}}	%for diagonal matrices

%%abstract algebras
\DeclareMathOperator{\ord}{\operatorname{ord}}
\DeclareMathOperator{\gr}{\operatorname{gr}}
\DeclareMathOperator{\Frac}{\operatorname{Frac}}

%%%symbols for basic geometries
\DeclareMathOperator{\vol}{\operatorname{vol}}	%volume
\DeclareMathOperator{\dist}{\operatorname{dist}}
\DeclareMathOperator{\supp}{\operatorname{supp}}

%%%symbols for category
%%names of categories
\DeclareMathOperator{\Mod}{\operatorname{Mod}}
\DeclareMathOperator{\Vect}{\operatorname{Vect}}


%%%symbols for homological algebras
\DeclareMathOperator{\Tor}{\operatorname{Tor}}
\DeclareMathOperator{\Ext}{\operatorname{Ext}}
\DeclareMathOperator{\gldim}{\operatorname{gl.dim}}
\DeclareMathOperator{\projdim}{\operatorname{proj.dim}}
\DeclareMathOperator{\injdim}{\operatorname{inj.dim}}
\DeclareMathOperator{\rad}{\operatorname{rad}}


%%%symbols for algebraic groups
\DeclareMathOperator{\GL}{\operatorname{GL}}
\DeclareMathOperator{\SL}{\operatorname{SL}}

%%%symbols for typical varieties
\DeclareMathOperator{\Gr}{\operatorname{Gr}}
\DeclareMathOperator{\Flag}{\operatorname{Flag}}

%%%symbols for basic algebraic geometry
\DeclareMathOperator{\Spec}{\operatorname{Spec}}
\DeclareMathOperator{\Coh}{\operatorname{Coh}}
\newcommand{\Dcoh}{\mathcal{D}_{\operatorname{Coh}}}%%%This one shows the difference between \DeclareMathOperator and \newcommand
\DeclareMathOperator{\Pic}{\operatorname{Pic}}
\DeclareMathOperator{\Jac}{\operatorname{Jac}}

%%%%%advanced symbols. Choose the part you need!

%%%symbols for algebraic representation theory
\DeclareMathOperator{\ind}{\operatorname{ind}}	%\ind(Q) means the set of  equivalence classes of finite dimensional indecomposable representations
\DeclareMathOperator{\Res}{\operatorname{Res}}
\DeclareMathOperator{\Ind}{\operatorname{Ind}}
\DeclareMathOperator{\cInd}{\operatorname{c-Ind}}


\DeclareMathOperator{\Rep}{\operatorname{Rep}}
\DeclareMathOperator{\rep}{\operatorname{rep}} %usually rep means the category of finite dimensional representations, while Rep means the category of representations.
\DeclareMathOperator{\Irr}{\operatorname{Irr}}
\DeclareMathOperator{\irr}{\operatorname{irr}}
\DeclareMathOperator{\Adm}{\operatorname{\Pi}}
\DeclareMathOperator{\Char}{\operatorname{Char}}
\DeclareMathOperator{\WDrep}{\operatorname{WDrep}}

%%%symbols for algebraic topology
\DeclareMathOperator{\EGG}{\operatorname{E}\!}
\DeclareMathOperator{\BGG}{\operatorname{B}\!}

\DeclareMathOperator{\chern}{\operatorname{ch}^{*}}
\DeclareMathOperator{\Td}{\operatorname{Td}}
\DeclareMathOperator{\AS}{\operatorname{AS}}	%Atiyah--Segal completion theorem 

%%%symbols for Auslander--Reiten theory 
\DeclareMathOperator{\Modup}{\overline{\operatorname{mod}}}
\DeclareMathOperator{\Moddown}{\underline{\operatorname{mod}}}
\DeclareMathOperator{\Homup}{\overline{\operatorname{Hom}}}
\DeclareMathOperator{\Homdown}{\underline{\operatorname{Hom}}}


%%%symbols for operad
\DeclareMathOperator{\Com}{\operatorname{\mathcal{C}om}}
\DeclareMathOperator{\Ass}{\operatorname{\mathcal{A}ss}}
\DeclareMathOperator{\Lie}{\operatorname{\mathcal{L}ie}}
\DeclareMathOperator{\calEnd}{\operatorname{\mathcal{E}nd}} %cal=\mathcal


%%%%%personal symbols. Use at your own risk!

%%%symbols only for master thesis
\DeclareMathOperator{\ptt}{\operatorname{par}}	%the partition map
\DeclareMathOperator{\str}{\operatorname{str}}	%strict case
\DeclareMathOperator{\RRep}{\widetilde{\operatorname{Rep}}}
\DeclareMathOperator{\Rpt}{\operatorname{R}}
\DeclareMathOperator{\Rptc}{\operatorname{\mathcal{R}}}
\DeclareMathOperator{\Spt}{\operatorname{S}}
\DeclareMathOperator{\Sptc}{\operatorname{\mathcal{S}}}
\DeclareMathOperator{\Kcurl}{\operatorname{\mathcal{K}}}
\DeclareMathOperator{\Hcurl}{\operatorname{\mathcal{H}}}
\DeclareMathOperator{\eu}{\operatorname{eu}}
\DeclareMathOperator{\Eu}{\operatorname{Eu}}
\DeclareMathOperator{\dimv}{\operatorname{\underline{\mathbf{dim}}}}
\DeclareMathOperator{\St}{\mathcal{Z}}

%%%%%symbols which haven't been classified. Add your own math operators here!


\DeclareMathOperator{\Modr}{\operatorname{-Mod}}

\DeclareMathOperator{\Quot}{\operatorname{Quot}}
\DeclareMathOperator{\Sch}{\operatorname{Sch}}
\DeclareMathOperator{\Hilb}{\operatorname{Hilb}}
\let\xlongequal\relax
\usepackage{extpfeil} %for longer arrows

%%%%%%%newcommand

%%%basic symbols
\newcommand{\norm}[1]{\Vert{#1}\Vert}

%%%symbols only for master thesis
\newcommand{\dimvec}[1]{\mathbf{#1}}
\newcommand{\abdimvec}[1]{|\dimvec{#1}|}
\newcommand{\ftdimvec}[1]{\underline{\dimvec{#1}}}

\newcommand{\absgp}[1]{\mathbb{#1}}
\newcommand{\WWd}{\absgp{W}_{\abdimvec{d}}}
\newcommand{\Wd}{W_{\dimvec{d}}}
\newcommand{\MinWd}{\operatorname{Min}(\absgp{W}_{\abdimvec{d}},W_{\dimvec{d}})}
\newcommand{\Compd}{\operatorname{Comp}_{\dimvec{d}}}
\newcommand{\Shuffled}{\operatorname{Shuffle}_{\dimvec{d}}}

\newcommand{\Omcell}{\Omega}
\newcommand{\OOmcell}{\boldsymbol{\Omega}}
\newcommand{\Vcell}{\mathcal{V}}
\newcommand{\VVcell}{\boldsymbol{\mathcal{V}}}
\newcommand{\Ocell}{\mathcal{O}}
\newcommand{\OOcell}{\boldsymbol{\mathcal{O}}}
\newcommand{\preimage}[1]{\widetilde{#1}}
\newcommand{\orde}{\operatorname{ord}_e}
\newcommand{\fakestar}{*}

%as the subscription of Hom
\newcommand{\Alggp}{\text{-Alg gp}}

%as automorphic representations
\newcommand{\Acusp}{\mathcal{A}_{\operatorname{cusp}}}




%%%%%%%%%%%%%%%%%%%% here we make some blocks for special features. 

%%%% todo notes %%%%
\usepackage[colorinlistoftodos,textsize=footnotesize]{todonotes}
\setlength{\marginparwidth}{2.5cm}
\newcommand{\leftnote}[1]{\reversemarginpar\marginnote{\footnotesize #1}}
\newcommand{\rightnote}[1]{\normalmarginpar\marginnote{\footnotesize #1}\reversemarginpar}









%%%%%%%%%%%%%%%%%%%% here we make some global settings. Understand everything here before you make a document!

\usepackage[a4paper,left=3cm,right=3cm,bottom=4cm]{geometry}
\usepackage{indentfirst}	% Indent first paragraph after section header

\setcounter{tocdepth}{2}


%https://latexref.xyz/_005cparindent-_0026-_005cparskip.html
\setlength{\parindent}{15pt}	
\setlength{\parskip}{0pt plus1pt}

%\setlength\intextsep{0cm}
%\setlength\textfloatsep{0cm}
\def\arraystretch{1}
%\setcounter{secnumdepth}{3}

\allowdisplaybreaks


\begin{document}

% The beginning depends on the documentclass. Rewrite this part if you use different documentclass!
\date{\today}

\title
{\LaTeX\;Template
}
\author{Xiaoxiang Zhou}
\address{Institut für Mathematik\\
Humboldt-Universität zu Berlin\\
Berlin, 12489\\ Germany\\} 
\email{email:xiaoxiang.zhou@hu-berlin.de}


\maketitle
\tableofcontents


\subsection{Quot scheme}
We refer to \href{https://en.wikipedia.org/wiki/Quot_scheme}{wiki} and \href{https://gauss.math.yale.edu/~il282/Siddharth_S16.pdf}{Venkatesh's lecture notes}.

The Quot scheme is a relative version of the Grassmannian. Instead of parameterizing subvector spaces, it parameterizes quotient sheaves of a fixed sheaf.

\begin{eg}[Quot scheme $\Quot_{\mathcal{E}}$]
In this example, the field $\kappa$ can be generalized to a Noetherian base scheme $S_0$.

For a scheme $X/\kappa$ of finite type and $\mathcal{E} \in \Coh(X)$, we define a moduli problem for $(S,\pi_S: S \longrightarrow \Spec \kappa) \in \Ob(\Sch_{\kappa})$:

$$\mathcal{A}_S:=\left\{(\mathcal{F},\pi)  \;\middle|\; \begin{aligned}
&\\[-5mm]
&\mathcal{F} \in \Coh(X_S) \text{ flat over $S$ }\\[-1mm]
&\supp(\mathcal{F})  \text{ proper over $S$ }\\[-1mm]
& \pi:\mathcal{E}_S \xtwoheadrightarrow{\hspace{2mm}}  \mathcal{F}
\end{aligned}
 \right\}$$
 
   $(\mathcal{F},\pi) \sim_S (\mathcal{F}',\pi')$ if there exists an isomorphism of vector bundles $\phi:\mathcal{F} \longrightarrow \mathcal{F}'$ such that $\pi'=\phi \circ \pi$.
   
   For a map $f:T \longrightarrow S$, the pullback $f^*$ is defined by
      $$f^*:\mathcal{A}_S \longrightarrow \mathcal{A}_T \qquad (\mathcal{F},\pi) \longmapsto \big((\Id \times f)^*\mathcal{F},f^* \pi:\mathcal{E}_T \twoheadrightarrow (\Id \times f)^*\mathcal{F}\big)$$
      
      The moduli functor defined by this extended moduli problem is representable, and we denote it by $\Quot_{\mathcal{E}}$.
\end{eg}

\begin{remark}
The requirements of flatness and properness ensure the well-definedness of the Hilbert polynomial $P_{\mathcal{F}}(t) \in \mathbb{Q}[t]$, defines as follows.

When a line bundle $\mathcal{L}$ over $X$ is fixed, take a closed point $s \in S$, then 
$$P_{\mathcal{F}}(m) = \chi(\mathcal{F}_s \otimes \mathcal{L}_s^{\otimes m}) = \sum_{i=0}^{\dim \mathcal{F}} (-1)^i h^i (X_s, \mathcal{F}_s \otimes \mathcal{L}_s^{\otimes m}).$$
 By flatness,  the Hilbert polynomial $P_{\mathcal{F}}(m)$ does not depend on the choice of $s \in S$. Furthermore, when $X$ is projective, one can take $\mathcal{L} = \mathcal{O}_X(1)$. This provides us a decomposition of $\Quot_{\mathcal{E}}$:
 $$\Quot_{\mathcal{E}} = \bigsqcup_{P \in \mathbb{Q}[t]} \Quot_{\mathcal{E}}^P.$$
\end{remark}

\begin{eg}
When $X= \Spec \kappa$, $\mathcal{E}= \kappa^n$, one gets
$$\mathcal{A}_S=\left\{(\mathcal{F},\pi)  \;\middle|\; \begin{aligned}
&\\[-5mm]
&\mathcal{F} \in \Coh(S) \text{ flat over $S$ }\\[-1mm]
& \pi:\mathcal{O}_S^{\oplus n} \xtwoheadrightarrow{\hspace{2mm}}  \mathcal{F}
\end{aligned}
 \right\}$$
 and $P_{\mathcal{F}}(t)= \chi (\mathcal{F}_s)$ is constant, indicating the rank of $\mathcal{F}$. Therefore,
 $$\Gr(k,n) = \Quot_{\kappa^n}^{k}, \qquad \mathbb{P}V^{\vee}= \Quot_V^{1}.$$
\end{eg}

\begin{eg}
In this example, the base scheme $S_0$ is $X$, and $\mathcal{E} \in \Coh(X)$ is locally free. In this setting, the moduli problem for ??? is given by
$$\mathcal{A}_S=\left\{(\mathcal{F},\pi)  \;\middle|\; \begin{aligned}
&\\[-5mm]
&\mathcal{F} \in \Coh(S) \text{ flat over $S$ }\\[-1mm]
& \pi:\pi_S^*\mathcal{E} \xtwoheadrightarrow{\hspace{2mm}}  \mathcal{F}
\end{aligned}
 \right\}$$
 Since any finitely generated flat module over a commutative local, Noetherian ring is free (see \href{https://math.stackexchange.com/questions/1812584/finitely-generated-flat-modules-over-a-commutative-local-noetherian-ring-are-f}{SE1812584}),
 $\mathcal{F}$ is locally free of rank $r$, and $P_{\mathcal{F}}(t)=r$. Therefore,
 $$\mathbb{P}\mathcal{E} = \Quot_{\mathcal{E}}^{1} \text{ (over $X/X$)}.$$
\end{eg}

\begin{remark}
The tangent space and smoothness of $\Quot_{\mathcal{E}}$ can be described locally, just like $T_V \Gr(k,n) = \Hom_{\mathbb{C}}(V, \mathbb{C}^n/V)$, see \cite{bibid} for more details.
\end{remark}

\subsection{Hilbert scheme}
Since we already defined $\Quot_{\mathcal{E}}$, the Hilbert scheme is just a special case:
$$\Hilb_X:= \Quot_{\mathcal{O}_X} \qquad \Hilb_X^P:= \Quot_{\mathcal{O}_X}^P$$
where the moduli problem is given by

$$\mathcal{A}_S=\left\{(\mathcal{F},\pi)  \;\middle|\; \begin{aligned}
&\\[-5mm]
&\mathcal{F} \in \Coh(X_S) \text{ flat over $S$ }\\[-1mm]
&\supp(\mathcal{F})  \text{ proper over $S$ }\\[-1mm]
& \pi:\mathcal{O}_{X_S} \xtwoheadrightarrow{\hspace{2mm}}  \mathcal{F}
\end{aligned}
 \right\}$$
 
 In this case, $\mathcal{F} \cong i_* \mathcal{O}_Z$ for some subvariety (maybe non-reduced) of $X_S$ (of Hilbert polynomial $P$).
 
\begin{eg}
The Hilbert scheme of $k$ points on $X$ is denoted by $$X^{[k]}:= \Hilb_X^k= \Quot_{\mathcal{O}_X}^k.$$
It is birational equivalent to $X^k/S_k$.
\end{eg} 

\begin{eg}
Recall that $P_{\mathbb{P}^m}(t) = \binom{m+t}{m}$. For any closed integral subvariety $X \subset \mathbb{P}^n$, the Fano variety of $m$-planes $\mathbb{P}^m$ in $X$ is given by
$$F(X,m):= \Hilb_X^{P_{\mathbb{P}^m}}= \Quot_{\mathcal{O}_X}^{P_{\mathbb{P}^m}}.$$
\end{eg} 
\begin{eg}
Recall that $P_{Z}(t) = \binom{n+t}{n}- \binom{n+t-d}{n}$ for a degree $d$ hypersurface $Z$ in $\mathbb{P}^n$.
The moduli space of degree $d$ hypersurfaces in $\mathbb{P}^n$ is given by
$$\Hilb_{\mathbb{P}^n}^{P_{Z}}= \Quot_{\mathcal{O}_{\mathbb{P}^n}}^{P_{Z}} \cong \mathbb{P}\left(\rule{0mm}{3.2mm} \Gamma \left(\rule{0mm}{3mm} \mathbb{P}^n, \mathcal{O}(d) \right) \right) \cong \mathbb{P}^{\binom{n+d}{d}-1}.$$
\end{eg} 
\bibliographystyle{plain}
\bibliography{reference}
\end{document}