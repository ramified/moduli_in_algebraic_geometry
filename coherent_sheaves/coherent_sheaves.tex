
\documentclass[UTF8]{amsart}
%Typical documenttypes: article/book
%some examples:
%\documentclass[reqno,11pt]{book}   %%%for books
%\documentclass[]{minimal}			%%%for Minimal Working Example


%for beamers, you have to change a lot. Especially, remove the package enumitem!!!



%%%%%%%%%%%%%%%%%%%% setting for fast compiling

%\special{dvipdfmx:config z 0}		% no compression

\includeonly{chapters/chapter9}		% In practice, use an empty document called "chapter9"	% usually for printing books






%%%%%%%%%%%%%%%%%%%% here we include packages

%%%basic packages for math articles
\usepackage{amssymb}
\usepackage{amsthm}
\usepackage{amsmath}
\usepackage{amsfonts}
\usepackage[shortlabels]{enumitem}	% It supersedes both enumerate and mdwlist. The package option shortlabels is included to configure the labels like in enumerate.

%%%packages for special symbols
\usepackage{pifont}					% Access to PostScript standard Symbol and Dingbats fonts
\usepackage{wasysym}				% additional characters
\usepackage{bm}						% bold fonts: \bm{...}
\usepackage{extarrows}				% may be replaced by tikz-cd
%\usepackage{unicode-math}			% unicode maths for math fonts, now I don't know how to include it
%\usepackage{ctex}					% Chinese characters, huge difference.


%%%basic packages for fancy electronic documents
\usepackage[colorlinks]{hyperref}
\usepackage[table,hyperref]{xcolor} 			% before tikz-cd. 
%\usepackage[table,hyperref,monochrome]{xcolor}	% disable colored output (black and white)

%%%packages for figures and tables (general setting)
\usepackage{float}				%Improved interface for floating objects
\usepackage{caption,subcaption}
\usepackage{adjustbox}			% for me it is usually used in tables 
\usepackage{stackengine}		%baseline changes

%%%packages for commutative diagrams
\usepackage{tikz-cd}
\usepackage{quiver}			% see https://q.uiver.app/

%%%packages for pictures
\usepackage[width=0.5,tiewidth=0.7]{strands}
\usepackage{graphicx}			% Enhanced support for graphics

%%%packages for tables and general settings
\usepackage{array}
\usepackage{makecell}
\usepackage{multicol}
\usepackage{multirow}
\usepackage{diagbox}
\usepackage{longtable}

%%%packages for ToC, LoF and LoT







 %https://tex.stackexchange.com/questions/58852/possible-incompatibility-with-enumitem










%%%%%%%%%%%%%%%%%%%% here we include theoremstyles

\numberwithin{equation}{section}

\theoremstyle{plain}
\newtheorem{theorem}{Theorem}[section]

\newtheorem{setting}[theorem]{Setting}
\newtheorem{definition}[theorem]{Definition}
\newtheorem{lemma}[theorem]{Lemma}
\newtheorem{proposition}[theorem]{Proposition}
\newtheorem{corollary}[theorem]{Corollary}
\newtheorem{conjecture}[theorem]{Conjecture}

\newtheorem{claim}[theorem]{Claim}
\newtheorem{eg}[theorem]{Example}
\newtheorem{ex}[theorem]{Exercise}
\newtheorem{fact}[theorem]{Fact}
\newtheorem{ques}[theorem]{Question}
\newtheorem{warning}[theorem]{Warning}



\newtheorem*{bbox}{Black box}
\newtheorem*{notation}{Conventions and Notations}


\numberwithin{equation}{section}


\theoremstyle{remark}

\newtheorem{remark}[theorem]{Remark}
\newtheorem*{remarks}{Remarks}

%%% for important theorems
%\newtheoremstyle{theoremletter}{4mm}{1mm}{\itshape}{ }{\bfseries}{}{ }{}
%\theoremstyle{theoremletter}
%\newtheorem{theoremA}{Theorem}
%\renewcommand{\thetheoremA}{A}
%\newtheorem{theoremB}{Theorem}
%\renewcommand{\thetheoremB}{B}







%%%%%%%%%%%%%%%%%%%% here we declare some symbols

%%%%%%%DeclareMathOperator
%see here for why newcommand is better for DeclareMathOperator: https://tex.stackexchange.com/questions/67506/newcommand-vs-declaremathoperator

%%%%%basic symbols. Keep them!

%%%symbols for sets and maps
\DeclareMathOperator{\pt}{\operatorname{pt}}	%points. Other possibilities are \{pt\}, \{*\}, pt, * ...
\DeclareMathOperator{\Id}{\operatorname{Id}}	%identity in groups.
\DeclareMathOperator{\Img}{\operatorname{Im}}

\DeclareMathOperator{\Ob}{\operatorname{Ob}}
\DeclareMathOperator{\Mor}{\operatorname{Mor}}	%difference of Mor and Hom: Hom is usually for abelian categories
\DeclareMathOperator{\Hom}{\operatorname{Hom}}	\DeclareMathOperator{\End}{\operatorname{End}}
\DeclareMathOperator{\Aut}{\operatorname{Aut}}

%%%symbols for linear algebras and 
%%linear algebras
\DeclareMathOperator{\tr}{\operatorname{tr}}
\DeclareMathOperator{\diag}{\operatorname{diag}}	%for diagonal matrices

%%abstract algebras
\DeclareMathOperator{\ord}{\operatorname{ord}}
\DeclareMathOperator{\gr}{\operatorname{gr}}
\DeclareMathOperator{\Frac}{\operatorname{Frac}}

%%%symbols for basic geometries
\DeclareMathOperator{\vol}{\operatorname{vol}}	%volume
\DeclareMathOperator{\dist}{\operatorname{dist}}
\DeclareMathOperator{\supp}{\operatorname{supp}}

%%%symbols for category
%%names of categories
\DeclareMathOperator{\Mod}{\operatorname{Mod}}
\DeclareMathOperator{\Vect}{\operatorname{Vect}}


%%%symbols for homological algebras
\DeclareMathOperator{\Tor}{\operatorname{Tor}}
\DeclareMathOperator{\Ext}{\operatorname{Ext}}
\DeclareMathOperator{\gldim}{\operatorname{gl.dim}}
\DeclareMathOperator{\projdim}{\operatorname{proj.dim}}
\DeclareMathOperator{\injdim}{\operatorname{inj.dim}}
\DeclareMathOperator{\rad}{\operatorname{rad}}


%%%symbols for algebraic groups
\DeclareMathOperator{\GL}{\operatorname{GL}}
\DeclareMathOperator{\SL}{\operatorname{SL}}

%%%symbols for typical varieties
\DeclareMathOperator{\Gr}{\operatorname{Gr}}
\DeclareMathOperator{\Flag}{\operatorname{Flag}}

%%%symbols for basic algebraic geometry
\DeclareMathOperator{\Spec}{\operatorname{Spec}}
\DeclareMathOperator{\Coh}{\operatorname{Coh}}
\newcommand{\Dcoh}{\mathcal{D}_{\operatorname{Coh}}}%%%This one shows the difference between \DeclareMathOperator and \newcommand
\DeclareMathOperator{\Pic}{\operatorname{Pic}}
\DeclareMathOperator{\Jac}{\operatorname{Jac}}

%%%%%advanced symbols. Choose the part you need!

%%%symbols for algebraic representation theory
\DeclareMathOperator{\ind}{\operatorname{ind}}	%\ind(Q) means the set of  equivalence classes of finite dimensional indecomposable representations
\DeclareMathOperator{\Res}{\operatorname{Res}}
\DeclareMathOperator{\Ind}{\operatorname{Ind}}
\DeclareMathOperator{\cInd}{\operatorname{c-Ind}}


\DeclareMathOperator{\Rep}{\operatorname{Rep}}
\DeclareMathOperator{\rep}{\operatorname{rep}} %usually rep means the category of finite dimensional representations, while Rep means the category of representations.
\DeclareMathOperator{\Irr}{\operatorname{Irr}}
\DeclareMathOperator{\irr}{\operatorname{irr}}
\DeclareMathOperator{\Adm}{\operatorname{\Pi}}
\DeclareMathOperator{\Char}{\operatorname{Char}}
\DeclareMathOperator{\WDrep}{\operatorname{WDrep}}

%%%symbols for algebraic topology
\DeclareMathOperator{\EGG}{\operatorname{E}\!}
\DeclareMathOperator{\BGG}{\operatorname{B}\!}

\DeclareMathOperator{\chern}{\operatorname{ch}^{*}}
\DeclareMathOperator{\Td}{\operatorname{Td}}
\DeclareMathOperator{\AS}{\operatorname{AS}}	%Atiyah--Segal completion theorem 

%%%symbols for Auslander--Reiten theory 
\DeclareMathOperator{\Modup}{\overline{\operatorname{mod}}}
\DeclareMathOperator{\Moddown}{\underline{\operatorname{mod}}}
\DeclareMathOperator{\Homup}{\overline{\operatorname{Hom}}}
\DeclareMathOperator{\Homdown}{\underline{\operatorname{Hom}}}


%%%symbols for operad
\DeclareMathOperator{\Com}{\operatorname{\mathcal{C}om}}
\DeclareMathOperator{\Ass}{\operatorname{\mathcal{A}ss}}
\DeclareMathOperator{\Lie}{\operatorname{\mathcal{L}ie}}
\DeclareMathOperator{\calEnd}{\operatorname{\mathcal{E}nd}} %cal=\mathcal


%%%%%personal symbols. Use at your own risk!

%%%symbols only for master thesis
\DeclareMathOperator{\ptt}{\operatorname{par}}	%the partition map
\DeclareMathOperator{\str}{\operatorname{str}}	%strict case
\DeclareMathOperator{\RRep}{\widetilde{\operatorname{Rep}}}
\DeclareMathOperator{\Rpt}{\operatorname{R}}
\DeclareMathOperator{\Rptc}{\operatorname{\mathcal{R}}}
\DeclareMathOperator{\Spt}{\operatorname{S}}
\DeclareMathOperator{\Sptc}{\operatorname{\mathcal{S}}}
\DeclareMathOperator{\Kcurl}{\operatorname{\mathcal{K}}}
\DeclareMathOperator{\Hcurl}{\operatorname{\mathcal{H}}}
\DeclareMathOperator{\eu}{\operatorname{eu}}
\DeclareMathOperator{\Eu}{\operatorname{Eu}}
\DeclareMathOperator{\dimv}{\operatorname{\underline{\mathbf{dim}}}}
\DeclareMathOperator{\St}{\mathcal{Z}}

%%%%%symbols which haven't been classified. Add your own math operators here!


\DeclareMathOperator{\Modr}{\operatorname{-Mod}}
\DeclareMathOperator{\Stab}{\operatorname{Stab}}




%%%%%%%newcommand

%%%basic symbols
\newcommand{\norm}[1]{\Vert{#1}\Vert}

%%%symbols only for master thesis
\newcommand{\dimvec}[1]{\mathbf{#1}}
\newcommand{\abdimvec}[1]{|\dimvec{#1}|}
\newcommand{\ftdimvec}[1]{\underline{\dimvec{#1}}}

\newcommand{\absgp}[1]{\mathbb{#1}}
\newcommand{\WWd}{\absgp{W}_{\abdimvec{d}}}
\newcommand{\Wd}{W_{\dimvec{d}}}
\newcommand{\MinWd}{\operatorname{Min}(\absgp{W}_{\abdimvec{d}},W_{\dimvec{d}})}
\newcommand{\Compd}{\operatorname{Comp}_{\dimvec{d}}}
\newcommand{\Shuffled}{\operatorname{Shuffle}_{\dimvec{d}}}

\newcommand{\Omcell}{\Omega}
\newcommand{\OOmcell}{\boldsymbol{\Omega}}
\newcommand{\Vcell}{\mathcal{V}}
\newcommand{\VVcell}{\boldsymbol{\mathcal{V}}}
\newcommand{\Ocell}{\mathcal{O}}
\newcommand{\OOcell}{\boldsymbol{\mathcal{O}}}
\newcommand{\preimage}[1]{\widetilde{#1}}
\newcommand{\orde}{\operatorname{ord}_e}
\newcommand{\fakestar}{*}

%as the subscription of Hom
\newcommand{\Alggp}{\text{-Alg gp}}

%as automorphic representations
\newcommand{\Acusp}{\mathcal{A}_{\operatorname{cusp}}}




%%%%%%%%%%%%%%%%%%%% here we make some blocks for special features. 

%%%% todo notes %%%%
\usepackage[colorinlistoftodos,textsize=footnotesize]{todonotes}
\setlength{\marginparwidth}{2.5cm}
\newcommand{\leftnote}[1]{\reversemarginpar\marginnote{\footnotesize #1}}
\newcommand{\rightnote}[1]{\normalmarginpar\marginnote{\footnotesize #1}\reversemarginpar}









%%%%%%%%%%%%%%%%%%%% here we make some global settings. Understand everything here before you make a document!

\usepackage[a4paper,left=3cm,right=3cm,bottom=4cm]{geometry}
\usepackage{indentfirst}	% Indent first paragraph after section header

\setcounter{tocdepth}{2}


%https://latexref.xyz/_005cparindent-_0026-_005cparskip.html
\setlength{\parindent}{15pt}	
\setlength{\parskip}{0pt plus1pt}

%\setlength\intextsep{0cm}
%\setlength\textfloatsep{0cm}
\def\arraystretch{1}
%\setcounter{secnumdepth}{3}

\allowdisplaybreaks


\begin{document}

% The beginning depends on the documentclass. Rewrite this part if you use different documentclass!
\date{\today}

\title
{Student seminar: Moduli of vector bundles
}
\author{Xiaoxiang Zhou}
\address{Institut für Mathematik\\
Humboldt-Universität zu Berlin\\
Berlin, 12489\\ Germany\\} 
\email{email:xiaoxiang.zhou@hu-berlin.de}


\maketitle
%\tableofcontents
\subsection*{Talk 1: Introduction}

\subsection*{Talk 2: Explicit constructions of semistable bundles on elliptic curves}
%\addcontentsline{toc}{section}{Talk 2: Explicit constructions of semistable bundles on elliptic curves}

In this talk, we introduce semistability of vector bundle on curves, and provide first non-trivial example of semistable vector bundles.

\begin{itemize}
\item Define slope stability \cite[Definition 2.3]{MS19}. State some basic properties, e.g., \cite[Exercise 2.4]{MS19}, \cite[14.1]{Potier97}.
\item Recall the classification of vector bundles on $\mathbb{P}^1$, and determine when they are (semi)stable.
\item Show the picture of Ford circles. Fix an elliptic curve $(E, p_0)$. For each rational number $\mu = \frac{d}{r} > 0$, construct a stable vector bundle $V_\mu$ of rank $r$ and degree $d$ such that $\det V_\mu \cong \mathcal{O}_E(dp_0)$. \cite[14.3]{Potier97}
\item Verify the stability of $V_\mu$ by induction. Shows that
$$
\dim \Hom (V_{\mu_1}, V_{\mu_2}) =\dim \Ext (V_{\mu_2}, V_{\mu_1}) = \begin{cases}
d_2r_1-d_1r_2, & \mu_1 < \mu_2 \\
1, & \mu_1 = \mu_2 \\
0, & \mu_1 > \mu_2. \\
\end{cases}
$$
Discuss further properties, including \cite[Corollary 14.11]{Potier97}. In particular, describe $H^{\bullet}(V_\mu)$.
\item For each $r \geq 1$ and $\mathcal{L} \in \Pic^0(E)$, construct a semistable vector bundle $V_{r, \mathcal{L}}$ of rank $r$ with $\det V_{r, \mathcal{L}} \cong \mathcal{L}$.
\item Conclude the talk by stating \cite[Example 2.7]{MS19} as a theorem.

\end{itemize}

\subsection*{Talk 3: Fourier--Mukai transform on elliptic curve}

The goal is to complete the classification of vector bundles on elliptic curves via the Fourier--Mukai transform.

\begin{itemize}
\item Define Fourier--Mukai transform $\Phi_K$ \cite[Chapter 11]{Potier97}.
\item Describe $\Phi_K \circ \Phi_L$ and the right adjoint of $\Phi_K$. Show that if $A, B$ are abelian varieties of dimension $n$ and $K \in \Pic(A \times B)$ is a line bundle, then both adjoints of $\Phi_K$ are given by $\Phi_{K^{-1}[n]}$.
\item Sketch the proof of \cite[Theorem 11.4]{Potier97} in the case $A = B$ is an elliptic curve and $S = \{*\}$, i.e., show that ...

\noindent \hspace{-8mm}Assume $A$ is an abelian variety from now on.

\item Define the Poincaré line bundle $\mathcal{P} \in \Pic(A \times \hat{A})$ and recall its basic properties.
\item Set $\mathcal{S} = \mathcal{S}_A = \Phi_{\mathcal{P}}$. List key properties of $\mathcal{S}$, e.g., \cite[(11.3.1)–(11.3.4), Theorem 11.6]{Potier97}, and the convolution identity
$$\mathcal{S}(\mathcal{F}) \otimes \mathcal{S}(\mathcal{G}) \cong \mathcal{S}(\mathcal{F}*\mathcal{G}) \qquad \text{\cite[p141]{Potier97}}$$

\item  $\mathcal{S}$ induces equivalences between subcategories of $D^b(A)$. Mention \cite[Proposition 11.8, Lemma 14.6, Theorem 14.7]{Potier97}, and explain how they yield a classification of vector bundles on elliptic curves.
\item Describe $\mathcal{S}(V_{r,\mathcal{L}})$ explicitly.
\item Fix a non-degenerate line bundle $\mathcal{L} \in \Pic(A)$. Discuss the $\mathrm{SL}_2(\mathbb{Z}) \rtimes \mathbb{Z}$ action on $D^b(A)$ and describe $\gamma(\mathcal{F})$ for special $\gamma \in \mathrm{SL}_2(\mathbb{Z})$ and $\mathcal{F} \in D^b(A)$.
\end{itemize}
If time is short, the speaker can focus on the elliptic curve case and skip all technical proofs.


\subsection*{Talk 4: Vector bundles on curves of genus $\geqslant 2$}\cite[2.3 \& 2.5]{MS19}
In this talk, we describe the moduli space of vector bundles over a curve of genus $\geqslant 2$, focusing on some special cases \cite[Examples 2.18–2.20]{MS19}.

\begin{itemize}
\item Begin by defining the moduli functor \cite[Definition 2.9]{MS19}, and mention its representability \cite[Theorem 2.10]{MS19} (you can also put mention it in the end of the talk).

\noindent \hspace{-8mm} Now fix a curve $C$ of genus $2$.
\item Explain why $M_C(2, \mathcal{O}_C) \cong \mathbb{P}^3$.
\item Sketch why $M_C(2, \mathcal{L}) \cong Q_1 \cap Q_2 \subset \mathbb{P}^5$, where $\mathcal{L}$ is a line bundle of degree $1$. Discuss the connection to semiorthogonal decompositions.
\item If time permits, outline the relation between $M_C(3, \mathcal{O}_C)$ and the Coble cubic hypersurface.
\end{itemize}

\subsection*{Talk 5: Semistable sheaves of degree $0$}\cite[2.4]{MS19}
This talk is centered on \cite[Theorem 2.14]{MS19} and its surrounding results. I regard it as the coherent counterpart of the Riemann--Hilbert correspondence.

\begin{itemize}
\item Present and prove \cite[Theorem 2.14]{MS19}.
\item Illustrate \cite[Theorem 2.14]{MS19} explicitly in the cases of elliptic curves and curves of genus $2$.
\item Briefly discuss the generalization in [NS65, Theorem 2].
\item  Discuss equivalent notions of stability for curves.
\end{itemize}

\subsection*{Talk 6: Stability manifold of $\mathbb{P}^1$}
This talk introduces Bridgeland stability conditions and discusses $\Stab(\mathbb{P}^1)$ in detail. The standard reference is [...]; you may also consult my notes.

\begin{itemize}
\item Define (locally finite) stability conditions on a triangulated category $\mathcal{T}$ and denote the space by $\Stab(\mathcal{T})$.
\item Show that the usual slope stability on $\Coh(\mathbb{P}^1)$ defines a stability condition $(Z_0, \mathcal{P}_0) \in \Stab(\mathbb{P}^1)$.
\item Discuss the $\mathbb{C}$ action on $\Stab(\mathcal{T})$. In the case of $\mathbb{P}^1$, restrict to the orbit of $(Z_0, \mathcal{P}_0)$ and describe how the heart changes. Note that semistability of objects remains unchanged under this action.
\item Classify stability conditions where all line bundles and torsion sheaves are semistable. Describe the $\mathbb{Z}$ action via tensoring with $\mathcal{O}(1)$, and the locus where $\mathcal{O}$ is semistable but not stable.
\item State Lemma 3.1(d).\footnote{Bonus point if you prove this in your notes!} Give a description of all remaining stability conditions.
\item Define walls in $\Stab(\mathbb{P}^1)$ and explain wall-crossing behavior.
\item If time permits, briefly discuss stability conditions on other curves.
\end{itemize}

\subsection*{Talk 7: Equivariant vector bundles on Grassmannian}

\subsection*{Talk 8: Chern class}

\subsection*{Talk 9: Exceptional vector bundles on $\mathbb{P}^2$}

\subsection*{Talk 10: Stbility manifold of surfaces}

\subsection*{Talk 11: vector bundles on K3 surfaces}

\subsection*{Talk 12: vector bundles on threefolds}
%remember to protect the uppercase of people's name and LaTeX symbols

\bibliographystyle{plain}
\bibliography{reference}
\end{document}