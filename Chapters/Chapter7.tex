\section{After FOAG: the future plan}
When I finished reading the book \cite{FOAG}, I felt the confidence of understanding everything in algebraic geometry field. However, I felt soon so confused and helpless, because of the superabundance of topics and articles which are not linearly developed, and they intersected with each other. I had totally no idea what goal to set and what to read. Is it still possible to organize all the (relatively advanced) basics in algebraic geometry, just like Prof. Vakil did in \cite{FOAG}?

This survey is one important part of the whole plan, which aimed to fill in everything well-known to experts but unknown for me. Chinese discussion can be found \href{https://www.zhihu.com/question/318263266}{here}.

\begin{itemize}
\item A series of classes, such as complex algebraic surfaces, toric varieties, Abelian varieties, finite group schemes, ...
\item Moduli theory. It is this survey, even though we missed still a lot:
\begin{itemize}
\item Modular curve and Schimura variety
\item Fermat's last theorem
\end{itemize}
I wonder If I would say anything about these in this survey.
\item Cohomology theory. In my mind derived category as well as six-functors formalism are the basic tool boxes, and Prof. Scholze's picture concludes the cohomology theories in a magical way. In particular, we need to show:
\begin{itemize}
\item Étale cohomology
\item The proof of Weil's conjectures
\end{itemize}
\item Use scheme-analogic models to solve problems.
\begin{itemize}
\item Berkovich spaces, p-adic spaces, and formal schemes.
\item Theory of height. The typical example is Faltings's theorem. 
\end{itemize}
\item Anything with analysis. Index theory and symplectic geometry can be related.
\item Anything with representation theory and number theory. Langlands program is for me the central goal. Class field theory and Iwasawa theory can be also relative topics.
\item Anything with combination. For example, the dessin d'enfant, knot, number field, ...
\end{itemize}