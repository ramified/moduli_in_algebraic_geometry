\section{Basic object}
In this section, we present some algebraic geometric objects which can be viewed as moduli.

Here is a picture showing the relationships of these objects:
% https://tikzcd.yichuanshen.de/#N4Igdg9gJgpgziAXAbVABwnAlgFyxMJZABgBpiBdUkANwEMAbAVxiRAB12BbOnACwBGA4AAUAvgDUQY0uky58hFGQCMVWoxZtOPfkNFidvPgGNGwAGJjpskBmx4CRFaTXV6zVog7sA4gCcACn9SCQBKGzkHRWdSACZ1Dy1vTgBFJggcSLt5RyUSeMTNLx8ACSwGAWz7BScUOML3Yu12CwY6AHMoQPDpdRgoDvgiUAAzfwguJDIQHAgkFRkxianEGbmkOKWQccnN6g3EAGZt3dWXWfnj05WkABYDq5PbM-3LpABWMQoxIA
\[\begin{tikzcd}[row sep={12mm,between origins}, column sep={15mm,between origins}]
\mathbb{P}V \arrow[d] \arrow[rd] &                                 &           \\
\mathbb{P}\mathcal{F} \arrow[rd] & {\Gr(r,V)} \arrow[d] \arrow[rd] &           \\
\Hilb \arrow[r]                  & \Quot                           & \Flagd(V)
\end{tikzcd}\]
\subsection{Projective space}
We begin with a basic extended moduli problem, and then gradually make some variations.
\begin{eg}[Moduli of line bundle with base-point-free sections is represented by $\mathbb{P}_k^n$]
Fix $n \geqslant 0$, we define a moduli problem:
$$\mathcal{A}_S:=\left\{(\mathcal{L},s_0,\ldots,s_n)  \;\middle|\; \begin{aligned}
&\\[-5mm]
&\mathcal{L} \in \Pic(S)\\[-1mm]
& s_i \in \Gamma(S,\mathcal{L}) \text{ with no common zero }
\end{aligned}
 \right\}$$
 
  $(\mathcal{L},s_0,\ldots,s_n) \sim_S (\mathcal{L}',s_0',\ldots,s_n')$ if there exists an isomorphism of line bundles $\phi:\mathcal{L} \longrightarrow \mathcal{L}'$ such that $\phi(S):\Gamma(S,\mathcal{L}) \longrightarrow \Gamma(S,\mathcal{L}')$ sends $s_i$ to $s_i'$.
  
  For a map $f:T \longrightarrow S$, the pullback $f^*$ is defined by
     $$f^*:\mathcal{A}_S \longrightarrow \mathcal{A}_T \qquad (\mathcal{L},s_0,\ldots,s_n) \longmapsto (f^*\mathcal{L},f^*s_0,\ldots,f^*s_n)$$
     
By \cite[15.3.F, 16.4.1]{FOAG}, the moduli functor defined by this extended moduli problem is represented by $\mathbb{P}_k^n$.
\end{eg}
We also have the coordinate-free version.
\begin{eg}[{Coordinate-free projective space $\mathbb{P}V^{\vee}=\Proj(\Sym^{\bullet} V)$, see \cite[4.5.12]{FOAG}}]
Fix a $k$-vector spcace $V$ of finite dimension. We define a moduli problem:
$$\mathcal{A}_S:=\left\{(\mathcal{L},\lambda)  \;\middle|\; \begin{aligned}
&\\[-5mm]
&\mathcal{L} \in \Pic(S)\\[-1mm]
& \lambda:V \longrightarrow \Gamma(S,\mathcal{L}) \text{ is base-point-free }
\end{aligned}
 \right\}$$
 
   $(\mathcal{L},\lambda) \sim_S (\mathcal{L}',\lambda')$ if there exists an isomorphism of line bundles $\phi:\mathcal{L} \longrightarrow \mathcal{L}'$ such that $\lambda'=\phi(S) \circ \lambda$.
   
   For a map $f:T \longrightarrow S$, the pullback $f^*$ is defined by
      $$f^*:\mathcal{A}_S \longrightarrow \mathcal{A}_T \qquad (\mathcal{L},\lambda) \longmapsto \big(f^*\mathcal{L},f^*\circ \lambda:V \rightarrow \Gamma(T,f^*\mathcal{L})\big)$$
      
      By \cite[16.4.E]{FOAG}, the moduli functor defined by this extended moduli problem is represented by $\mathbb{P}V^{\vee}$.
\end{eg}
Now we generalize it to the projective bundle, for this we should fix a scheme $X \in \Ob(\Schk)$ and a locally free coherent sheaf $\mathcal{F} \in \Coh(X)$, and consider the moduli problem in the category $\SchX$ of locally Noetherian schemes over $X$, rather than $\Schk$.

\begin{eg}[{Projective bundle\footnote{There is an notational abuse in \cite{FOAG}. From my personal point of view, it's better to replace $\mathbb{P}\mathcal{F}$ with $\mathbb{P}\mathcal{F}^{\vee}$ or $\mathbb{P}V^{\vee}$ with $\mathbb{P}V$ to make symbols consistent. } $\mathbb{P}\mathcal{F}=\Proj(\Sym^{\bullet} \mathcal{F})$, see \cite[17.2.3]{FOAG}}]\hspace{1cm}

For $(S,\pi_S: S \longrightarrow X) \in \Ob(\SchX)$, we define a moduli problem:
$$\mathcal{A}_S:=\left\{(\mathcal{L},\lambda)  \;\middle|\; \begin{aligned}
&\\[-5mm]
&\mathcal{L} \in \Pic(S)\\[-1mm]
& \lambda:\pi_S^* \mathcal{F} \xtwoheadrightarrow{\hspace{2mm}}  \mathcal{L}
\end{aligned}
 \right\}$$
 
   $(\mathcal{L},\lambda) \sim_S (\mathcal{L}',\lambda')$ if there exists an isomorphism of line bundles $\phi:\mathcal{L} \longrightarrow \mathcal{L}'$ such that $\lambda'=\phi \circ \lambda$.
   
   For a map $f:T \longrightarrow S$, the pullback $f^*$ is defined by
      $$f^*:\mathcal{A}_S \longrightarrow \mathcal{A}_T \qquad (\mathcal{L},\lambda) \longmapsto \big(f^*\mathcal{L},f^* \lambda:\pi_T^*\mathcal{F} \twoheadrightarrow f^*\mathcal{L}\big)$$
      
      By \cite[Proposition 7.12]{hartshorne2013algebraic}, the moduli functor defined by this extended moduli problem is represented by $\mathbb{P}\mathcal{F}$.
\end{eg}

Finally, there is also "another" natural moduli problem of $\mathbb{P}_k^n$, see \cite[Example 2.4]{modulicurve}. For the convinience of comparison with Grassmannian, we exhibit it and make some small variations here.\footnote{The reason of the variation is already explained in \cite[16.7, page 442]{FOAG}.}

\begin{eg}[{Moduli of lines through the origin in $\mathbb{A}_k^{n+1}$ is represented by $\mathbb{P}_k^n$}]
For $n \geqslant 0$, we define a moduli problem:
$$\mathcal{A}_S:=\left\{(\mathcal{L},\pi)  \;\middle|\; \begin{aligned}
&\\[-5mm]
&\mathcal{L} \in \Pic(S)\\[-1mm]
& \pi:\mathcal{O}_S^{\oplus n+1} \xtwoheadrightarrow{\hspace{2mm}}  \mathcal{L}
\end{aligned}
 \right\}$$
 
   $(\mathcal{L},\pi) \sim_S (\mathcal{L}',\pi')$ if there exists an isomorphism of line bundles $\phi:\mathcal{L} \longrightarrow \mathcal{L}'$ such that $\pi'=\phi \circ \pi$.
   
   For a map $f:T \longrightarrow S$, the pullback $f^*$ is defined by
      $$f^*:\mathcal{A}_S \longrightarrow \mathcal{A}_T \qquad (\mathcal{L},\pi) \longmapsto \big(f^*\mathcal{L},f^* \pi:\mathcal{O}_T^{\oplus n+1} \twoheadrightarrow f^*\mathcal{L}\big)$$
      
      By \cite[Proposition 7.12]{hartshorne2013algebraic}, the moduli functor defined by this extended moduli problem is represented by $\mathbb{P}\mathcal{F}$.
\end{eg}
\subsection{Grassmannian}
It's well-written in \cite[16.7]{FOAG}. We just exhibit(copy) the moduli problem and make a short remark about the existence proof (prove the representability without explicite construction of the scheme)
\begin{eg}[{Grassmannian $\Gr(k,n)$}]
For $n \geqslant k \geqslant 0$, we define a moduli problem:
$$\mathcal{A}_S:=\left\{(\mathcal{F},\pi)  \;\middle|\; \begin{aligned}
&\\[-5mm]
&\mathcal{F} \in \Coh(S) \text{ locally free of rank $k$ }\\[-1mm]
& \pi:\mathcal{O}_S^{\oplus n} \xtwoheadrightarrow{\hspace{2mm}}  \mathcal{F}
\end{aligned}
 \right\}$$
 
   $(\mathcal{F},\pi) \sim_S (\mathcal{F}',\pi')$ if there exists an isomorphism of line bundles $\phi:\mathcal{F} \longrightarrow \mathcal{F}'$ such that $\pi'=\phi \circ \pi$.
   
   For a map $f:T \longrightarrow S$, the pullback $f^*$ is defined by
      $$f^*:\mathcal{A}_S \longrightarrow \mathcal{A}_T \qquad (\mathcal{F},\pi) \longmapsto \big(f^*\mathcal{F},f^* \pi:\mathcal{O}_T^{\oplus n} \twoheadrightarrow f^*\mathcal{F}\big)$$
      
      By \cite[16.7, page 442-443]{FOAG}, the moduli functor defined by this extended moduli problem is representable, and we denote it by $\Gr(k,n)$.
\end{eg}
\begin{remark}
The idea of the proof comes from \cite[9.1.I]{FOAG}. First we prove it to be the Zariski sheaf, then we cover it with open subfunctors that are representable.
\end{remark}

\begin{remark}
It may help to extend $(\mathcal{F},\pi)$ in $\mathcal{A}_S$ to a short exact sequence
% https://q.uiver.app/#q=WzAsNSxbMCwwLCIwIl0sWzEsMCwiXFxrZXIgXFxwaSJdLFsyLDAsIlxcbWF0aGNhbHtPfV9TXntcXG9wbHVzIG59Il0sWzMsMCwiXFxtYXRoY2Fse0Z9Il0sWzQsMCwiMCJdLFswLDFdLFsxLDJdLFsyLDNdLFszLDRdXQ==
\[\begin{tikzcd}[column sep=2.25em]
	0 & {\ker \pi} & {\mathcal{O}_S^{\oplus n}} & {\mathcal{F}} & 0
	\arrow[from=1-1, to=1-2]
	\arrow[from=1-2, to=1-3]
	\arrow[from=1-3, to=1-4]
	\arrow[from=1-4, to=1-5]
\end{tikzcd}\]
and visualize it as the pullback of vector bundles over $\Gr(k,n)$.

\textbf{From map to sheaf:}
In each point $[V] \in \Gr(k,n)$, we have a short exact sequence:
\[\begin{tikzcd}[column sep=2.25em]
	0 & V & {k^n} & {k^n/V} & 0. 
	\arrow[from=1-1, to=1-2]
	\arrow[from=1-2, to=1-3]
	\arrow[from=1-3, to=1-4]
	\arrow[from=1-4, to=1-5]
\end{tikzcd}\]
Gluing fibers, one can get a short exact sequence of vector bundles over $\Gr(k,n)$:
\begin{equation}\label{eq:canonical_ses_over_Gr}
\begin{tikzcd}[column sep=2.25em]
	0 & {\mathcal{S}} & {\mathcal{O}_{\Gr(k,n)}^{\oplus n}} & {\mathcal{Q}} & 0.
	\arrow[from=1-1, to=1-2]
	\arrow[from=1-2, to=1-3]
	\arrow[from=1-3, to=1-4]
	\arrow[from=1-4, to=1-5]
\end{tikzcd}
\end{equation}
Here, $\mathcal{S}$ is the tautological vector bundle over $\Gr(k,n)$, and ${\mathcal{Q}}$ is the quotient vector bundle. 

When $k=1$, $\mathcal{S} = \mathcal{O}(-1)$. In this case, people working in algebraic geometry are not quite satisfied. They want $\mathcal{F}$ to be the pullback of $\mathcal{O}(1)$, and they want nice sections in $\mathcal{O}(1)$ coming from the map $\mathcal{O}^{\oplus n} \longrightarrow \mathcal{O}(1)$.

Luckily, by applying the functor $(-)^{\vee}= \Hom_{\mathcal{O}}(-,\mathcal{O})$, \eqref{eq:canonical_ses_over_Gr} is equivalent to (since all terms are vector bundles)
\begin{equation}\label{eq:canonical_ses_over_Gr_dual}
\begin{tikzcd}[column sep=2.25em]
	0 & {\mathcal{Q}^{\vee}} & {\mathcal{O}_{\Gr(k,n)}^{\oplus n}} & {\mathcal{S}^{\vee}} & 0.
	\arrow[from=1-1, to=1-2]
	\arrow[from=1-2, to=1-3]
	\arrow[from=1-3, to=1-4]
	\arrow[from=1-4, to=1-5]
\end{tikzcd}
\end{equation}
Pulling back through the map $S \longrightarrow \Gr(k,n)$, one gets
\[\begin{tikzcd}[column sep=2.25em]
	0 & {\ker \pi} & {\mathcal{O}_S^{\oplus n}} & {\mathcal{F}} & 0.
	\arrow[from=1-1, to=1-2]
	\arrow[from=1-2, to=1-3]
	\arrow[from=1-3, to=1-4]
	\arrow[from=1-4, to=1-5]
\end{tikzcd}\]
By taking stalks, this short exact sequence degenerates to
\[\begin{tikzcd}[column sep=2.25em]
	0 & {(k^n/V)^*} & {(k^n)^*} & V^* & 0. 
	\arrow[from=1-1, to=1-2]
	\arrow[from=1-2, to=1-3]
	\arrow[from=1-3, to=1-4]
	\arrow[from=1-4, to=1-5]
\end{tikzcd}\]

\textbf{From sheaf to map:} A vector bundle $\mathcal{E}$ over $S$ of rank $k$ with $n$ sections provides us with a short exact sequence 
\[\begin{tikzcd}[column sep=2.25em]
	0 & {\ker \pi} & {\mathcal{O}_S^{\oplus n}} & {\mathcal{E}} & 0
	\arrow[from=1-1, to=1-2]
	\arrow[from=1-2, to=1-3]
	\arrow["\pi", from=1-3, to=1-4]
	\arrow[from=1-4, to=1-5]
\end{tikzcd}\]
whose geometry looks like
\[\begin{tikzcd}[column sep=2.25em]
	0 & {E} & {\underline{k}^n} & {\underline{k}^n/E} & 0.
	\arrow[from=1-1, to=1-2]
	\arrow[from=1-2, to=1-3]
	\arrow["\pi", from=1-3, to=1-4]
	\arrow[from=1-4, to=1-5]
\end{tikzcd}\]
\end{remark}
\begin{eg}
For a proper smooth curve $\mathcal{C}$ of genus $g>1$, the Gaussian map
$$\phi: \mathcal{C} \longrightarrow \mathbb{P}^{g-1} \qquad p \longmapsto [T_p\mathcal{C}]$$
is not induced by the sheaf $\mathcal{T}_{\mathcal{C}}$, but by $\omega_{\mathcal{C}}=\mathcal{T}_{\mathcal{C}}^{\vee}$. One can check that $c_1(\omega_{\mathcal{C}})=\phi^* c_1(\mathcal{O}(1))$.
\end{eg}
\subsection{Flag variety, partial flag variety}
\subsection{Hilbert scheme}
\subsection{Quot scheme}
The moduli space of projective hypersurfaces is a special case for this.
\subsection{Misc}The representable functor is also used to construct the fibered product of schemes, see \cite[9.1.6-7]{FOAG} for more details.

We've already met the idea of moduli in plane geometry. For example, fix two points $A$, $B$ on the plane, we wanted to find the set of points $C$ such that $\triangle ABC$ is a right triangle (resp. an isosceles triangle). We used the term ``locus" in junior high school.

\begin{figure}[th]
	\begin{minipage}[t]{.36\textwidth}
		\centering

\begin{tikzpicture}[thick,help lines/.style={thin,draw=black}]
\def\A{\textcolor{input}{$A$}} \def\B{\textcolor{input}{$B$}}
\def\C{\textcolor{output}{$C$}} 
\colorlet{input}{black} \colorlet{output}{red!70!black}
\colorlet{triangle}{orange}
\coordinate [label=left:\A] (A) at ($ (0,0) $);
\coordinate [label=right:\B] (B) at ($ (1.75,0.3)$);
\coordinate (F) at ($ (A) ! 0.5 ! (B) $);
\coordinate (G) at ($ (A) ! 2 ! (B) $);
\node [name path=D,help lines,draw] (D) at (F) [circle through=(B)] {};
\draw [help lines]($ (A) ! -1.2! 90:(B) $)-- ($ (A) ! 1.2! 90:(B) $);
\draw [help lines]($ (B) ! -1.2! 90:(A) $)-- ($ (B) ! 1.2! 90:(A) $);
%\draw [output] (A) -- (C) -- (B);
\foreach \point in {A,B}
\fill [white, draw={black},thin] (\point) circle (2pt);
%\begin{pgfonlayer}{background}
%\fill[triangle!80] (A) -- (C) -- (B) -- cycle;
%\end{pgfonlayer}
\end{tikzpicture}
	\end{minipage}
	\begin{minipage}[t]{.36\textwidth}
\begin{tikzpicture}[thick,help lines/.style={thin,draw=black}]
\def\A{\textcolor{input}{$A$}} \def\B{\textcolor{input}{$B$}}
\def\C{\textcolor{output}{$C$}} 
\colorlet{input}{black} \colorlet{output}{red!70!black}
\colorlet{triangle}{orange}
\coordinate [label=left:\A] (A) at ($ (0,0) $);
\coordinate [label=right:\B] (B) at ($ (1.75,0.3)$);
\coordinate (F) at ($ (A) ! -1 ! (B) $);
\coordinate (G) at ($ (A) ! 2 ! (B) $);
\coordinate (H) at ($ (A) ! 0.5 ! (B) $);
\node [name path=D,help lines,draw] (D) at (A) [circle through=(B)] {};
\node [name path=E,help lines,draw] (E) at (B) [circle through=(A)] {};
\path [name intersections={of=D and E,by={C,C'}}];
\draw [help lines]($ (C) ! -0.2 ! (C') $)-- ($ (C) ! 1.2 ! (C') $);
%\draw [output] (A) -- (C) -- (B);
\foreach \point in {A,B,F,G,H}
\fill [white, draw={black},thin] (\point) circle (2pt);
%\begin{pgfonlayer}{background}
%\fill[triangle!80] (A) -- (C) -- (B) -- cycle;
%\end{pgfonlayer}
\end{tikzpicture}
	\end{minipage}
\caption{``moduli space'' in Euclidean geometry}
\end{figure}

There are still quite a lot of interesting examples of moduli spaces not presented here, see examples in \cite[0.1.1, 0.2]{alper2021introduction}.
