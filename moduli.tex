
\documentclass[reqno,11pt]{amsart}

%\usepackage{color,graphicx}
%\usepackage{mathrsfs,amsbsy}
\usepackage{amssymb}
\usepackage{amsmath}
\usepackage{amsfonts}
\usepackage{booktabs}
\usepackage{bm}
\usepackage{graphicx}
\usepackage{amsthm}
\usepackage{physics}
\usepackage{enumerate}
\usepackage[mathscr]{eucal}
\usepackage{float}
\usepackage{mathrsfs}
\usepackage{multicol}
\usepackage{multirow}
\usepackage{stackengine}
\usepackage[all,pdf]{xy}
\usepackage[a4paper,left=3cm,right=3cm]{geometry}
\usepackage[table,xcdraw]{xcolor} % before tikz-cd
\usepackage{tikz-cd}
\usepackage{hyperref}
\hypersetup{
colorlinks=true,
linkcolor=blue,
urlcolor=blue,
}
\usepackage{quiver} 
\usepackage{changepage}
\usepackage{extpfeil} %for longer arrows
%\usepackage[notcite,notref]{showkeys}
\usepackage{tikz}
\usetikzlibrary {backgrounds,calc,intersections,through}
\usetikzlibrary {positioning,shapes.misc}
% showkeys  make label explicit on the paper


\definecolor{ashgrey}{rgb}{0.7, 0.75, 0.71}
\makeatletter
\@namedef{subjclassname@2010}{%
  \textup{2010} Mathematics Subject Classification}
\makeatother

\numberwithin{equation}{section}

\theoremstyle{plain}
\newtheorem{theorem}{Theorem}[section]
\newtheorem{lemma}[theorem]{Lemma}
\newtheorem{proposition}[theorem]{Proposition}
\newtheorem{corollary}[theorem]{Corollary}
\newtheorem{claim}[theorem]{Claim}
\newtheorem{defn}[theorem]{Definition}
\newtheorem{ques}[theorem]{Question}
\newtheorem*{fact}{Facts}
\newtheorem{eg}[theorem]{Example}

\theoremstyle{plain}
\newtheorem{thmsub}{Theorem}[subsection]
\newtheorem{lemmasub}[thmsub]{Lemma}
\newtheorem{corollarysub}[thmsub]{Corollary}
\newtheorem{propositionsub}[thmsub]{Proposition}
\newtheorem{defnsub}[thmsub]{Definition}

\numberwithin{equation}{section}


\theoremstyle{remark}

\newtheorem{remark}[theorem]{Remark}
\newtheorem{remarks}{Remarks}
\newtheorem*{remarkstar}{Remark}
\newtheorem*{problem}{Question}
\newtheorem*{answer}{Answer}
\newtheorem*{easyex}{Easy exercise}
\newtheorem{warning}{Warning}
%\renewcommand\thefootnote{\fnsymbol{footnote}}
%dont use number as footnote symbol, use this command to change
\DeclareMathOperator{\chara}{\operatorname{char}}
\DeclareMathOperator{\supp}{supp}
\DeclareMathOperator{\dist}{dist}
\DeclareMathOperator{\vol}{vol}
\DeclareMathOperator{\diag}{diag}
%\DeclareMathOperator{\tr}{tr}
\DeclareMathOperator{\Img}{\operatorname{Im}}
\DeclareMathOperator{\Id}{\operatorname{Id}}
\DeclareMathOperator{\Rep}{\operatorname{Rep}}
\DeclareMathOperator{\Mod}{\operatorname{Mod}}
\DeclareMathOperator{\Hom}{\operatorname{Hom}}
\DeclareMathOperator{\Ext}{\operatorname{Ext}}
\DeclareMathOperator{\gldim}{\operatorname{gl.dim}}
\DeclareMathOperator{\projdim}{\operatorname{proj.dim}}
\DeclareMathOperator{\injdim}{\operatorname{inj.dim}}
\DeclareMathOperator{\dimv}{\operatorname{\underline{\mathbf{dim}}}}
\DeclareMathOperator{\SL}{\operatorname{SL}}
\DeclareMathOperator{\PGL}{\operatorname{PGL}}
\DeclareMathOperator{\Flagd}{\operatorname{Flag}_{d}}
\DeclareMathOperator{\Flagdstr}{\operatorname{Flag}_{d,str}}
\newcommand{\Gr}{\operatorname{Gr}}
\newcommand{\Grr}{\operatorname{Gr}}
\newcommand{\Grq}{\operatorname{Gr}^{KQ}}
\newcommand{\Flag}[1]{\operatorname{Flag}_{#1}}
\newcommand{\Flagstr}[1]{\operatorname{Flag}_{#1,str}}
\newcommand{\dimvec}[1]{#1}
\newcommand{\rderiv}[2]{\mathrm{R}^{#1}\! {#2}}
\newcommand{\ord}{\operatorname{ord}}
\newcommand{\orde}{\operatorname{ord}_e }
\newcommand{\representation}[2]{\genfrac{}{}{0pt}{3}{\phantom{000}#2\phantom{00}}{#1}}
\newcommand{\Spec}{\operatorname{Spec}}
\newcommand{\Proj}{\operatorname{Proj}}
\newcommand{\Spf}{\operatorname{Spf}}
\newcommand{\Pic}{\operatorname{Pic}}
\newcommand{\Coh}{\operatorname{Coh}}
\newcommand{\Psh}{\operatorname{Psh}}
\newcommand{\Sch}{\operatorname{Sch}}
\newcommand{\Schk}{\operatorname{Sch}_{\kappa}}
\newcommand{\SchX}{\operatorname{Sch}_{X}}
\newcommand{\Set}{\operatorname{Set}}
\newcommand{\Ob}{\operatorname{Ob}}
\newcommand{\Mor}{\operatorname{Mor}}
\newcommand{\Hilb}{\operatorname{Hilb}}
\newcommand{\Quot}{\operatorname{Quot}}
\newcommand{\Sym}{\operatorname{Sym}}


\setlength\intextsep{0cm}
\setlength\textfloatsep{0cm}


\makeatletter
\newcommand\subsectiondisappear{\@startsection{subsubsection}{2}%
  \z@{.5\linespacing\@plus.7\linespacing}{-.5em}%
  {\normalfont\bfseries}}
\makeatother

\makeatletter
%Table of Contents
\setcounter{tocdepth}{3}

% Add bold to \section titles in ToC and remove . after numbers
\renewcommand{\tocsection}[3]{%
  \indentlabel{\@ifnotempty{#2}{\bfseries\ignorespaces#1 #2\quad}}\bfseries#3}
% Remove . after numbers in \subsection
\renewcommand{\tocsubsection}[3]{%
  \indentlabel{\@ifnotempty{#2}{\ignorespaces#1 #2\quad}}#3}
%\let\tocsubsubsection\tocsubsection% Update for \subsubsection
%...

\newcommand\@dotsep{4.5}
\def\@tocline#1#2#3#4#5#6#7{\relax
  \ifnum #1>\c@tocdepth % then omit
  \else
    \par \addpenalty\@secpenalty\addvspace{#2}%
    \begingroup \hyphenpenalty\@M
    \@ifempty{#4}{%
      \@tempdima\csname r@tocindent\number#1\endcsname\relax
    }{%
      \@tempdima#4\relax
    }%
    \parindent\z@ \leftskip#3\relax \advance\leftskip\@tempdima\relax
    \rightskip\@pnumwidth plus1em \parfillskip-\@pnumwidth
    #5\leavevmode\hskip-\@tempdima{#6}\nobreak
    \leaders\hbox{$\m@th\mkern \@dotsep mu\hbox{.}\mkern \@dotsep mu$}\hfill
    \nobreak
    \hbox to\@pnumwidth{\@tocpagenum{\ifnum#1=1\bfseries\fi#7}}\par% <-- \bfseries for \section page
    \nobreak
    \endgroup
  \fi}
\AtBeginDocument{%
\expandafter\renewcommand\csname r@tocindent0\endcsname{0pt}
}
\def\l@subsection{\@tocline{2}{0pt}{2.5pc}{5pc}{}}

\makeatother
%Table of contents in amsart: https://tex.stackexchange.com/questions/322268/table-of-contents-amsart

\stackMath
\def\longsubset{\mathrel{
  \stackinset{l}{7pt}{c}{}{\wrule[-6.3pt]{10pt}{.34pt}}{
  \stackinset{l}{7pt}{c}{}{\wrule[6pt]{10pt}{.34pt}}{
 \subset\hspace{15pt}}}
}}
\newcommand\wrule[3][0pt]{\textcolor{black}{\rule[#1]{#2}{#3}}}

%\includeonly{Chapters/Chapter7}
\begin{document}
\date{}

\title
{Moduli in Algebraic Geometry}


\author{Xiaoxiang Zhou}
\address{School of Mathematical Sciences\\
University of Bonn\\
Bonn, 53115\\ Germany\\} 
\email{email:xx352229@mail.ustc.edu.cn}





\begin{abstract}
In this personal survey, we conclude the definitions of moduli functors in the algebraic geometry. Most of the results are in the black box, so it's very possible that they're wrong. And also I'm not responsible for the completeness of the whole theory. I make no claim to originality. However, I'm still happy to improve this document, and make it better over time.

\end{abstract}

\setcounter{tocdepth}{2}
\maketitle
\tableofcontents 

\include{Chapters/Chapter1}
\section{Basic object}
In this section, we present some algebraic geometric objects which can be viewed as moduli.

Here is a picture showing the relationships of these objects:
% https://tikzcd.yichuanshen.de/#N4Igdg9gJgpgziAXAbVABwnAlgFyxMJZABgBpiBdUkANwEMAbAVxiRAB12BbOnACwBGA4AAUAvgDUQY0uky58hFGQCMVWoxZtOPfkNFidvPgGNGwAGJjpskBmx4CRFaTXV6zVog7sA4gCcACn9SCQBKGzkHRWdSACZ1Dy1vTgBFJggcSLt5RyUSeMTNLx8ACSwGAWz7BScUOML3Yu12CwY6AHMoQPDpdRgoDvgiUAAzfwguJDIQHAgkFRkxianEGbmkOKWQccnN6g3EAGZt3dWXWfnj05WkABYDq5PbM-3LpABWMQoxIA
\[\begin{tikzcd}[row sep={12mm,between origins}, column sep={15mm,between origins}]
\mathbb{P}V \arrow[d] \arrow[rd] &                                 &           \\
\mathbb{P}\mathcal{E} \arrow[rd] & {\Gr(r,V)} \arrow[d] \arrow[rd] &           \\
\Hilb_X \arrow[r]                  & \Quot_{\mathcal{E}}                           & \Flagd(V)
\end{tikzcd}\]
\subsection{Projective space}
We begin with a basic extended moduli problem, and then gradually make some variations.
\begin{eg}[Moduli of line bundle with base-point-free sections is represented by $\mathbb{P}_k^n$]
Fix $n \geqslant 0$, we define a moduli problem:
$$\mathcal{A}_S:=\left\{(\mathcal{L},s_0,\ldots,s_n)  \;\middle|\; \begin{aligned}
&\\[-5mm]
&\mathcal{L} \in \Pic(S)\\[-1mm]
& s_i \in \Gamma(S,\mathcal{L}) \text{ with no common zero }
\end{aligned}
 \right\}$$
 
  $(\mathcal{L},s_0,\ldots,s_n) \sim_S (\mathcal{L}',s_0',\ldots,s_n')$ if there exists an isomorphism of line bundles $\phi:\mathcal{L} \longrightarrow \mathcal{L}'$ such that $\phi(S):\Gamma(S,\mathcal{L}) \longrightarrow \Gamma(S,\mathcal{L}')$ sends $s_i$ to $s_i'$.
  
  For a map $f:T \longrightarrow S$, the pullback $f^*$ is defined by
     $$f^*:\mathcal{A}_S \longrightarrow \mathcal{A}_T \qquad (\mathcal{L},s_0,\ldots,s_n) \longmapsto (f^*\mathcal{L},f^*s_0,\ldots,f^*s_n)$$
     
By \cite[15.3.F, 16.4.1]{FOAG}, the moduli functor defined by this extended moduli problem is represented by $\mathbb{P}_k^n$.
\end{eg}
We also have the coordinate-free version.
\begin{eg}[{Coordinate-free projective space $\mathbb{P}V^{\vee}=\Proj(\Sym^{\bullet} V)$, see \cite[4.5.12]{FOAG}}]
Fix a $k$-vector spcace $V$ of finite dimension. We define a moduli problem:
$$\mathcal{A}_S:=\left\{(\mathcal{L},\lambda)  \;\middle|\; \begin{aligned}
&\\[-5mm]
&\mathcal{L} \in \Pic(S)\\[-1mm]
& \lambda:V \longrightarrow \Gamma(S,\mathcal{L}) \text{ is base-point-free }
\end{aligned}
 \right\}$$
 
   $(\mathcal{L},\lambda) \sim_S (\mathcal{L}',\lambda')$ if there exists an isomorphism of line bundles $\phi:\mathcal{L} \longrightarrow \mathcal{L}'$ such that $\lambda'=\phi(S) \circ \lambda$.
   
   For a map $f:T \longrightarrow S$, the pullback $f^*$ is defined by
      $$f^*:\mathcal{A}_S \longrightarrow \mathcal{A}_T \qquad (\mathcal{L},\lambda) \longmapsto \big(f^*\mathcal{L},f^*\circ \lambda:V \rightarrow \Gamma(T,f^*\mathcal{L})\big)$$
      
      By \cite[16.4.E]{FOAG}, the moduli functor defined by this extended moduli problem is represented by $\mathbb{P}V^{\vee}$.
\end{eg}
Now we generalize it to the projective bundle, for this we should fix a scheme $X \in \Ob(\Schk)$ and a locally free coherent sheaf $\mathcal{E} \in \Coh(X)$, and consider the moduli problem in the category $\SchX$ of locally Noetherian schemes over $X$, rather than $\Schk$.

\begin{eg}[{Projective bundle\footnote{There is an notational abuse in \cite{FOAG}. From my personal point of view, it's better to replace $\mathbb{P}\mathcal{E}$ with $\mathbb{P}\mathcal{E}^{\vee}$ or $\mathbb{P}V^{\vee}$ with $\mathbb{P}V$ to make symbols consistent. } $\mathbb{P}\mathcal{E}=\Proj(\Sym^{\bullet} \mathcal{E})$, see \cite[17.2.3]{FOAG}}]\hspace{1cm}

For $(S,\pi_S: S \longrightarrow X) \in \Ob(\SchX)$, we define a moduli problem:
$$\mathcal{A}_S:=\left\{(\mathcal{L},\lambda)  \;\middle|\; \begin{aligned}
&\\[-5mm]
&\mathcal{L} \in \Pic(S)\\[-1mm]
& \lambda:\pi_S^* \mathcal{E} \xtwoheadrightarrow{\hspace{2mm}}  \mathcal{L}
\end{aligned}
 \right\}$$
 
   $(\mathcal{L},\lambda) \sim_S (\mathcal{L}',\lambda')$ if there exists an isomorphism of line bundles $\phi:\mathcal{L} \longrightarrow \mathcal{L}'$ such that $\lambda'=\phi \circ \lambda$.
   
   For a map $f:T \longrightarrow S$, the pullback $f^*$ is defined by
      $$f^*:\mathcal{A}_S \longrightarrow \mathcal{A}_T \qquad (\mathcal{L},\lambda) \longmapsto \big(f^*\mathcal{L},f^* \lambda:\pi_T^*\mathcal{E} \twoheadrightarrow f^*\mathcal{L}\big)$$
      
      By \cite[Proposition II.7.12]{hartshorne2013algebraic}, the moduli functor defined by this extended moduli problem is represented by $\mathbb{P}\mathcal{E}$ (over $X$).
\end{eg}

Finally, there is also "another" natural moduli problem of $\mathbb{P}_k^n$, see \cite[Example 2.4]{modulicurve}. For the convinience of comparison with Grassmannian, we exhibit it and make some small variations here.\footnote{The reason of the variation is already explained in \cite[16.7, page 442]{FOAG}.}

\begin{eg}[{Moduli of lines through the origin in $\mathbb{A}_k^{n+1}$ is represented by $\mathbb{P}_k^n$}]
For $n \geqslant 0$, we define a moduli problem:
$$\mathcal{A}_S:=\left\{(\mathcal{L},\pi)  \;\middle|\; \begin{aligned}
&\\[-5mm]
&\mathcal{L} \in \Pic(S)\\[-1mm]
& \pi:\mathcal{O}_S^{\oplus n+1} \xtwoheadrightarrow{\hspace{2mm}}  \mathcal{L}
\end{aligned}
 \right\}$$
 
   $(\mathcal{L},\pi) \sim_S (\mathcal{L}',\pi')$ if there exists an isomorphism of line bundles $\phi:\mathcal{L} \longrightarrow \mathcal{L}'$ such that $\pi'=\phi \circ \pi$.
   
   For a map $f:T \longrightarrow S$, the pullback $f^*$ is defined by
      $$f^*:\mathcal{A}_S \longrightarrow \mathcal{A}_T \qquad (\mathcal{L},\pi) \longmapsto \big(f^*\mathcal{L},f^* \pi:\mathcal{O}_T^{\oplus n+1} \twoheadrightarrow f^*\mathcal{L}\big)$$
\end{eg}
\subsection{Grassmannian}
It's well-written in \cite[16.7]{FOAG}. We just exhibit(copy) the moduli problem and make a short remark about the existence proof (prove the representability without explicite construction of the scheme)
\begin{eg}[{Grassmannian $\Gr(k,n)$}]
For $n \geqslant k \geqslant 0$, we define a moduli problem:
$$\mathcal{A}_S:=\left\{(\mathcal{F},\pi)  \;\middle|\; \begin{aligned}
&\\[-5mm]
&\mathcal{F} \in \Coh(S) \text{ locally free of rank $k$ }\\[-1mm]
& \pi:\mathcal{O}_S^{\oplus n} \xtwoheadrightarrow{\hspace{2mm}}  \mathcal{F}
\end{aligned}
 \right\}$$
 
   $(\mathcal{F},\pi) \sim_S (\mathcal{F}',\pi')$ if there exists an isomorphism of vector bundles $\phi:\mathcal{F} \longrightarrow \mathcal{F}'$ such that $\pi'=\phi \circ \pi$.
   
   For a map $f:T \longrightarrow S$, the pullback $f^*$ is defined by
      $$f^*:\mathcal{A}_S \longrightarrow \mathcal{A}_T \qquad (\mathcal{F},\pi) \longmapsto \big(f^*\mathcal{F},f^* \pi:\mathcal{O}_T^{\oplus n} \twoheadrightarrow f^*\mathcal{F}\big)$$
      
      By \cite[16.7, page 442-443]{FOAG}, the moduli functor defined by this extended moduli problem is representable, and we denote it by $\Gr(k,n)$.
\end{eg}
\begin{remark}
The idea of the proof comes from \cite[9.1.I]{FOAG}. First we prove it to be the Zariski sheaf, then we cover it with open subfunctors that are representable.
\end{remark}

\begin{remark}
It may help to extend $(\mathcal{F},\pi)$ in $\mathcal{A}_S$ to a short exact sequence
% https://q.uiver.app/#q=WzAsNSxbMCwwLCIwIl0sWzEsMCwiXFxrZXIgXFxwaSJdLFsyLDAsIlxcbWF0aGNhbHtPfV9TXntcXG9wbHVzIG59Il0sWzMsMCwiXFxtYXRoY2Fse0Z9Il0sWzQsMCwiMCJdLFswLDFdLFsxLDJdLFsyLDNdLFszLDRdXQ==
\[\begin{tikzcd}[column sep=2.25em]
	0 & {\ker \pi} & {\mathcal{O}_S^{\oplus n}} & {\mathcal{F}} & 0
	\arrow[from=1-1, to=1-2]
	\arrow[from=1-2, to=1-3]
	\arrow[from=1-3, to=1-4]
	\arrow[from=1-4, to=1-5]
\end{tikzcd}\]
and visualize it as the pullback of vector bundles over $\Gr(k,n)$.

\textbf{From map to sheaf:}
In each point $[V] \in \Gr(k,n)$, we have a short exact sequence:
\[\begin{tikzcd}[column sep=2.25em]
	0 & V & {k^n} & {k^n/V} & 0. 
	\arrow[from=1-1, to=1-2]
	\arrow[from=1-2, to=1-3]
	\arrow[from=1-3, to=1-4]
	\arrow[from=1-4, to=1-5]
\end{tikzcd}\]
Gluing fibers, one can get a short exact sequence of vector bundles over $\Gr(k,n)$:
\begin{equation}\label{eq:canonical_ses_over_Gr}
\begin{tikzcd}[column sep=2.25em]
	0 & {\mathcal{S}} & {\mathcal{O}_{\Gr(k,n)}^{\oplus n}} & {\mathcal{Q}} & 0.
	\arrow[from=1-1, to=1-2]
	\arrow[from=1-2, to=1-3]
	\arrow[from=1-3, to=1-4]
	\arrow[from=1-4, to=1-5]
\end{tikzcd}
\end{equation}
Here, $\mathcal{S}$ is the tautological vector bundle over $\Gr(k,n)$, and ${\mathcal{Q}}$ is the quotient vector bundle. 

When $k=1$, $\mathcal{S} = \mathcal{O}(-1)$. In this case, people working in algebraic geometry are not quite satisfied. They want $\mathcal{F}$ to be the pullback of $\mathcal{O}(1)$, and they want nice sections in $\mathcal{O}(1)$ coming from the map $\mathcal{O}^{\oplus n} \longrightarrow \mathcal{O}(1)$.

Luckily, by applying the functor $(-)^{\vee}= \Hom_{\mathcal{O}}(-,\mathcal{O})$, \eqref{eq:canonical_ses_over_Gr} is equivalent to (since all terms are vector bundles)
\begin{equation}\label{eq:canonical_ses_over_Gr_dual}
\begin{tikzcd}[column sep=2.25em]
	0 & {\mathcal{Q}^{\vee}} & {\mathcal{O}_{\Gr(k,n)}^{\oplus n}} & {\mathcal{S}^{\vee}} & 0.
	\arrow[from=1-1, to=1-2]
	\arrow[from=1-2, to=1-3]
	\arrow[from=1-3, to=1-4]
	\arrow[from=1-4, to=1-5]
\end{tikzcd}
\end{equation}
Pulling back through the map $S \longrightarrow \Gr(k,n)$, one gets
\[\begin{tikzcd}[column sep=2.25em]
	0 & {\ker \pi} & {\mathcal{O}_S^{\oplus n}} & {\mathcal{F}} & 0.
	\arrow[from=1-1, to=1-2]
	\arrow[from=1-2, to=1-3]
	\arrow[from=1-3, to=1-4]
	\arrow[from=1-4, to=1-5]
\end{tikzcd}\]
By taking stalks, this short exact sequence degenerates to
\[\begin{tikzcd}[column sep=2.25em]
	0 & {(k^n/V)^*} & {(k^n)^*} & V^* & 0. 
	\arrow[from=1-1, to=1-2]
	\arrow[from=1-2, to=1-3]
	\arrow[from=1-3, to=1-4]
	\arrow[from=1-4, to=1-5]
\end{tikzcd}\]

\textbf{From sheaf to map:} A vector bundle $\mathcal{E}$ over $S$ of rank $k$ with $n$ sections provides us with a short exact sequence 
\[\begin{tikzcd}[column sep=2.25em]
	0 & {\ker \pi} & {\mathcal{O}_S^{\oplus n}} & {\mathcal{E}} & 0
	\arrow[from=1-1, to=1-2]
	\arrow[from=1-2, to=1-3]
	\arrow["\pi", from=1-3, to=1-4]
	\arrow[from=1-4, to=1-5]
\end{tikzcd}\]
whose geometry looks like
\[\begin{tikzcd}[column sep=2.25em]
	0 & {E} & {\underline{k}^n} & {\underline{k}^n/E} & 0.
	\arrow[from=1-1, to=1-2]
	\arrow[from=1-2, to=1-3]
	\arrow["\pi", from=1-3, to=1-4]
	\arrow[from=1-4, to=1-5]
\end{tikzcd}\]
\end{remark}
\begin{eg}
For a proper smooth curve $\mathcal{C}$ of genus $g>1$, the Gaussian map
$$\phi: \mathcal{C} \longrightarrow \mathbb{P}^{g-1} \qquad p \longmapsto [T_p\mathcal{C}]$$
is not induced by the sheaf $\mathcal{T}_{\mathcal{C}}$, but by $\omega_{\mathcal{C}}=\mathcal{T}_{\mathcal{C}}^{\vee}$. One can check that $c_1(\omega_{\mathcal{C}})=\phi^* c_1(\mathcal{O}(1))$.
\end{eg}
\subsection{Flag variety, partial flag variety}
\subsection{Quot scheme}
We refer to \href{https://en.wikipedia.org/wiki/Quot_scheme}{wiki} and \href{https://gauss.math.yale.edu/~il282/Siddharth_S16.pdf}{Venkatesh's lecture notes}.

The Quot scheme is a relative version of the Grassmannian. Instead of parameterizing subvector spaces, it parameterizes quotient sheaves of a fixed sheaf.

\begin{eg}[Quot scheme $\Quot_{\mathcal{E}}$]
In this example, the field $\kappa$ can be generalized to a Noetherian base scheme $S_0$.

For a scheme $X/\kappa$ of finite type and $\mathcal{E} \in \Coh(X)$, we define a moduli problem for $(S,\pi_S: S \longrightarrow \Spec \kappa) \in \Ob(\Sch_{\kappa})$:

$$\mathcal{A}_S:=\left\{(\mathcal{F},\pi)  \;\middle|\; \begin{aligned}
&\\[-5mm]
&\mathcal{F} \in \Coh(X_S) \text{ flat over $S$ }\\[-1mm]
&\supp(\mathcal{F})  \text{ proper over $S$ }\\[-1mm]
& \pi:\mathcal{E}_S \xtwoheadrightarrow{\hspace{2mm}}  \mathcal{F}
\end{aligned}
 \right\}$$
 
   $(\mathcal{F},\pi) \sim_S (\mathcal{F}',\pi')$ if there exists an isomorphism of vector bundles $\phi:\mathcal{F} \longrightarrow \mathcal{F}'$ such that $\pi'=\phi \circ \pi$.
   
   For a map $f:T \longrightarrow S$, the pullback $f^*$ is defined by
      $$f^*:\mathcal{A}_S \longrightarrow \mathcal{A}_T \qquad (\mathcal{F},\pi) \longmapsto \big((\Id \times f)^*\mathcal{F},f^* \pi:\mathcal{E}_T \twoheadrightarrow (\Id \times f)^*\mathcal{F}\big)$$
      
      The moduli functor defined by this extended moduli problem is representable, and we denote it by $\Quot_{\mathcal{E}}$.
\end{eg}

\begin{remark}
The requirements of flatness and properness ensure the well-definedness of the Hilbert polynomial $P_{\mathcal{F}}(t) \in \mathbb{Q}[t]$, defines as follows.

When a line bundle $\mathcal{L}$ over $X$ is fixed, take a closed point $s \in S$, then 
$$P_{\mathcal{F}}(m) = \chi(\mathcal{F}_s \otimes \mathcal{L}_s^{\otimes m}) = \sum_{i=0}^{\dim \mathcal{F}} (-1)^i h^i (X_s, \mathcal{F}_s \otimes \mathcal{L}_s^{\otimes m}).$$
 By flatness,  the Hilbert polynomial $P_{\mathcal{F}}(m)$ does not depend on the choice of $s \in S$. Furthermore, when $X$ is projective, one can take $\mathcal{L} = \mathcal{O}_X(1)$. This provides us a decomposition of $\Quot_{\mathcal{E}}$:
 $$\Quot_{\mathcal{E}} = \bigsqcup_{P \in \mathbb{Q}[t]} \Quot_{\mathcal{E}}^P.$$
\end{remark}

\begin{eg}
When $X= \Spec \kappa$, $\mathcal{E}= \kappa^n$, one gets
$$\mathcal{A}_S=\left\{(\mathcal{F},\pi)  \;\middle|\; \begin{aligned}
&\\[-5mm]
&\mathcal{F} \in \Coh(S) \text{ flat over $S$ }\\[-1mm]
& \pi:\mathcal{O}_S^{\oplus n} \xtwoheadrightarrow{\hspace{2mm}}  \mathcal{F}
\end{aligned}
 \right\}$$
 and $P_{\mathcal{F}}(t)= \chi (\mathcal{F}_s)$ is constant, indicating the rank of $\mathcal{F}$. Therefore,
 $$\Gr(k,n) = \Quot_{\kappa^n}^{k}, \qquad \mathbb{P}V^{\vee}= \Quot_V^{1}.$$
\end{eg}

\begin{eg}
In this example, the base scheme $S_0$ is $X$, and $\mathcal{E} \in \Coh(X)$ is locally free. In this setting, the moduli problem for $(S,\pi_S: S \longrightarrow X) \in \Ob(\SchX)$ is given by
$$\mathcal{A}_S=\left\{(\mathcal{F},\pi)  \;\middle|\; \begin{aligned}
&\\[-5mm]
&\mathcal{F} \in \Coh(S) \text{ flat over $S$ }\\[-1mm]
& \pi:\pi_S^*\mathcal{E} \xtwoheadrightarrow{\hspace{2mm}}  \mathcal{F}
\end{aligned}
 \right\}$$
 Since any finitely generated flat module over a commutative local, Noetherian ring is free (see \href{https://math.stackexchange.com/questions/1812584/finitely-generated-flat-modules-over-a-commutative-local-noetherian-ring-are-f}{SE1812584}),
 $\mathcal{F}$ is locally free of rank $r$, and $P_{\mathcal{F}}(t)=r$. Therefore,
 $$\mathbb{P}\mathcal{E} = \Quot_{\mathcal{E}}^{1} \text{ (over $X/X$)}.$$
\end{eg}
\begin{remark}
The tangent space and smoothness of $\Quot_{\mathcal{E}}$ can be described locally, just like $T_V \Gr(k,n) = \Hom_{\mathbb{C}}(V, \mathbb{C}^n/V)$, see \cite[II.1.3]{Huy23} for more details.
\end{remark}

\subsection{Hilbert scheme}
Since we already defined $\Quot_{\mathcal{E}}$, the Hilbert scheme is just a special case:
$$\Hilb_X:= \Quot_{\mathcal{O}_X} \qquad \Hilb_X^P:= \Quot_{\mathcal{O}_X}^P$$
where the moduli problem is given by

$$\mathcal{A}_S=\left\{(\mathcal{F},\pi)  \;\middle|\; \begin{aligned}
&\\[-5mm]
&\mathcal{F} \in \Coh(X_S) \text{ flat over $S$ }\\[-1mm]
&\supp(\mathcal{F})  \text{ proper over $S$ }\\[-1mm]
& \pi:\mathcal{O}_{X_S} \xtwoheadrightarrow{\hspace{2mm}}  \mathcal{F}
\end{aligned}
 \right\}$$
 
 In this case, $\mathcal{F} \cong i_* \mathcal{O}_Z$ for some subvariety (maybe non-reduced) of $X_S$ (of Hilbert polynomial $P$).
 
\begin{eg}
The Hilbert scheme of $k$ points on $X$ is denoted by $$X^{[k]}:= \Hilb_X^k= \Quot_{\mathcal{O}_X}^k.$$
It has an irreducible component birational equivalent to $X^k/S_k$.\footnote{See \href{https://ravif.web.illinois.edu/exposition/seminar_talks/Components\%20of\%20Hilbert\%20schemes\%20of\%20points.pdf}{Components of Hilbert schemes of points} or \cite[Chapter 18]{MS05} for a survey.}
\end{eg} 

\begin{eg}
Recall that $P_{\mathbb{P}^m}(t) = \binom{m+t}{m}$. For any closed integral subvariety $X \subset \mathbb{P}^n$, the Fano variety of $m$-planes $\mathbb{P}^m$ in $X$ is given by
$$F(X,m):= \Hilb_X^{P_{\mathbb{P}^m}}= \Quot_{\mathcal{O}_X}^{P_{\mathbb{P}^m}}.$$
\end{eg} 
\begin{eg}
Recall that $P_{Z}(t) = \binom{n+t}{n}- \binom{n+t-d}{n}$ for a degree $d$ hypersurface $Z$ in $\mathbb{P}^n$.
The moduli space of degree $d$ hypersurfaces in $\mathbb{P}^n$ is given by
$$\Hilb_{\mathbb{P}^n}^{P_{Z}}= \Quot_{\mathcal{O}_{\mathbb{P}^n}}^{P_{Z}} \cong \mathbb{P}\left(\rule{0mm}{3.2mm} \Gamma \left(\rule{0mm}{3mm} \mathbb{P}^n, \mathcal{O}(d) \right) \right) \cong \mathbb{P}^{\binom{n+d}{d}-1}.$$
\end{eg} 




\subsection{Misc}The representable functor is also used to construct the fibered product of schemes, see \cite[9.1.6-7]{FOAG} for more details.

We've already met the idea of moduli in plane geometry. For example, fix two points $A$, $B$ on the plane, we wanted to find the set of points $C$ such that $\triangle ABC$ is a right triangle (resp. an isosceles triangle). We used the term ``locus" in junior high school.

\begin{figure}[th]
	\begin{minipage}[t]{.36\textwidth}
		\centering

\begin{tikzpicture}[thick,help lines/.style={thin,draw=black}]
\def\A{\textcolor{input}{$A$}} \def\B{\textcolor{input}{$B$}}
\def\C{\textcolor{output}{$C$}} 
\colorlet{input}{black} \colorlet{output}{red!70!black}
\colorlet{triangle}{orange}
\coordinate [label=left:\A] (A) at ($ (0,0) $);
\coordinate [label=right:\B] (B) at ($ (1.75,0.3)$);
\coordinate (F) at ($ (A) ! 0.5 ! (B) $);
\coordinate (G) at ($ (A) ! 2 ! (B) $);
\node [name path=D,help lines,draw] (D) at (F) [circle through=(B)] {};
\draw [help lines]($ (A) ! -1.2! 90:(B) $)-- ($ (A) ! 1.2! 90:(B) $);
\draw [help lines]($ (B) ! -1.2! 90:(A) $)-- ($ (B) ! 1.2! 90:(A) $);
%\draw [output] (A) -- (C) -- (B);
\foreach \point in {A,B}
\fill [white, draw={black},thin] (\point) circle (2pt);
%\begin{pgfonlayer}{background}
%\fill[triangle!80] (A) -- (C) -- (B) -- cycle;
%\end{pgfonlayer}
\end{tikzpicture}
	\end{minipage}
	\begin{minipage}[t]{.36\textwidth}
\begin{tikzpicture}[thick,help lines/.style={thin,draw=black}]
\def\A{\textcolor{input}{$A$}} \def\B{\textcolor{input}{$B$}}
\def\C{\textcolor{output}{$C$}} 
\colorlet{input}{black} \colorlet{output}{red!70!black}
\colorlet{triangle}{orange}
\coordinate [label=left:\A] (A) at ($ (0,0) $);
\coordinate [label=right:\B] (B) at ($ (1.75,0.3)$);
\coordinate (F) at ($ (A) ! -1 ! (B) $);
\coordinate (G) at ($ (A) ! 2 ! (B) $);
\coordinate (H) at ($ (A) ! 0.5 ! (B) $);
\node [name path=D,help lines,draw] (D) at (A) [circle through=(B)] {};
\node [name path=E,help lines,draw] (E) at (B) [circle through=(A)] {};
\path [name intersections={of=D and E,by={C,C'}}];
\draw [help lines]($ (C) ! -0.2 ! (C') $)-- ($ (C) ! 1.2 ! (C') $);
%\draw [output] (A) -- (C) -- (B);
\foreach \point in {A,B,F,G,H}
\fill [white, draw={black},thin] (\point) circle (2pt);
%\begin{pgfonlayer}{background}
%\fill[triangle!80] (A) -- (C) -- (B) -- cycle;
%\end{pgfonlayer}
\end{tikzpicture}
	\end{minipage}
\caption{``moduli space'' in Euclidean geometry}
\end{figure}

There are still quite a lot of interesting examples of moduli spaces not presented here, see examples in \cite[0.1.1, 0.2]{alper2021introduction}.

\include{Chapters/Chapter3}
\include{Chapters/Chapter4}
\include{Chapters/Chapter5}
\section{Moduli of vector bundle}
The course lecture note \href{https://userpage.fu-berlin.de/hoskins/moduli_and_GIT.html}{``Moduli and GIT"} would be a perfect survey to begin with. We also refer to \cite{huybrechts2010geometry}. It's not easy to read, but It's in some sense completed, and everybody refers it.

For a variant, you may get some informations of moduli of vector bundles over elliptic curve in \cite{atiyah1957vector} and $G$-bundles over elliptic curve in \cite{frăţilă2020revisiting}.

\textbf{Update on 2025.07.11:} Three years have gone by, and I’ve come to realize that I still have much to learn about this topic. I’ve fallen behind—there are already some striking classifications for particular cases, even though a complete picture is still missing. Below, I begin assembling a reference list of examples and tools for future learning. The main reference for this update is \cite{MS19}.

The following examples are already thoroughly studied and described in the literature:
\begin{itemize}
\item line bundles on any (smooth projective) variety: Picard variety;\footnote{Mention about ample cone and so on.}
\item vector bundles on $\mathbb{P}^1$;
\item (semi)stable vector bundles on elliptic curves, see \cite[Example 2.7]{MS19};
\item rank $2$ semistable vector bundles $E$ over a genus $2$ curve $C$, with either $\det E \cong \mathcal{O}_C$ or $\deg E = 1$, see \cite[Example 2.18,2.19]{MS19};
\item rank $3$ semistable vector bundles $E$ over a genus $2$ curve $C$, with $\det E \cong \mathcal{O}_C$, see \cite[Example 2.20]{MS19};
\item stable vector bundles $E$ on a curve $C$, with $\deg E = 0$, see \cite[Theorem 2.14]{MS19};
\item stable vector bundles $E$ on a curve $C$, see \cite[Theorem 2.10]{MS19};
\item exceptional coherent sheaves on $\mathbb{P}^2$, see \cite[1.1]{LZ19};
\item slope-semistable coherent sheaves on $\mathbb{P}^2$, see \cite[Theorem 1.8]{LZ19};
\item other surfaces: abelian/K3/Enrique, see \cite[p3]{MS19}.
\end{itemize}

The following invariants give rise to natural stratifications on moduli spaces of sheaves:
\begin{itemize}
\item Chern character: basic numerical invariants.
\item Sheaf cohomology: this is Brill--Noether theory.
\item Koszul cohomology, Betti table: this applies especially when the ambient space $X$ has a canonical embedding to $\mathbb{P}^n$. When the coherent sheaf is a quotient sheaf, their Koszul cohomology describe the equations defining the subvarieties.
\end{itemize}

A basic toolkit I need to learn for this area is:
\begin{itemize}
\item Bogomolov inequality;
\item GIT quotient;
\item vanishing results;
\item Gieseker stability and slope stability;
\item stability in abelian categories; Harder--Narasimhan filtration;
\item Bridgeland stability;
\item wall crossing techniques.
\end{itemize}


\include{Chapters/Chapter7}

\bibliographystyle{plain}
\bibliography{reference}





\end{document}

%流程图模板
%\usetikzlibrary {positioning,shapes.misc}
%\begin{tikzpicture}[node distance=5mm and 10mm, 
%initial/.style={
%color=red,
%}]
%\node (phiq2) [initial] {$\phi_q^2$ is surj};
%\node (phiq1) [below right=of phiq2] {$\phi_q^1$ is surj};
%\node (phiq0) [below right=of phiq1] {$\phi_q^0$ is surj};
%\node (phiq-1) [below right=of phiq0] {$\phi_q^{-1}$ is surj};
%\node (fiber2) [above right=of phiq2] {$\rderiv{2}{\pi_*}\mathcal{F}|_q$};
%\node (fiber1) [above right=of phiq1] {$\rderiv{1}{\pi_*}\mathcal{F}|_q$};
%\node (fiber0) [above right=of phiq0] {$\pi_*\mathcal{F}|_q$};
%\node (lf2) [right=of phiq2] {$\rderiv{2}{\pi_*}\mathcal{F}$ is l.f.};
%\node (lf1) [right=of phiq1] {$\rderiv{1}{\pi_*}\mathcal{F}$ is l.f.};
%\node (lf0) [right=of phiq0] {${\pi_*}\mathcal{F}$ is l.f.};
%\path ($ (lf2.south west) + (10mm,0) $) edge[commutative diagrams/Rightarrow, 2tail reversed] ($ (phiq1.north west) + (10mm,0) $);
%\path ($ (lf1.south west) + (10mm,0) $) edge[commutative diagrams/Rightarrow, 2tail reversed] ($ (phiq0.north west) + (10mm,0) $);
%\path ($ (lf0.south west) + (10mm,0) $) edge[commutative diagrams/Rightarrow, 2tail reversed] ($ (phiq-1.north west) + (10mm,0) $);
%\draw [->]
%(phiq2.east)
%-- ($ (phiq2.east) + (8mm,0) $)
%|- (fiber2.west);
%\draw [->]
%($ (phiq2.east) + (8mm,0) $)
%|- ($ (lf2.south west)!0.5!(phiq1.north west) + (8mm,0)$);
%\draw [->]
%(phiq1.east)
%-- ($ (phiq1.east) + (8mm,0) $)
%|- (fiber1.west);
%\draw [->]
%($ (phiq1.east) + (8mm,0) $)
%|- ($ (lf1.south west)!0.5!(phiq0.north west) + (8mm,0)$);
%\draw [->]
%(phiq0.east)
%-- ($ (phiq0.east) + (8mm,0) $)
%|- (fiber0.west);
%\draw [->]
%($ (phiq0.east) + (8mm,0) $)
%|- ($ (lf0.south west)!0.5!(phiq-1.north west) + (8mm,0)$);
%\end{tikzpicture}


